%% $Id: tutor.tex 2702 2014-09-18 13:43:39Z schwicht $

\documentclass[12pt]{amsart}
\usepackage{amssymb,enumerate}
\usepackage[backref]{hyperref}

\newcommand{\allnc}{\forall^{\mathrm{nc}}} %no computational content
\newcommand{\ex}{\exists} %constructive
\newcommand{\excl}{\tilde{\exists}} %classical, with logical falsum
\newcommand{\exu}{\exists^{\mathrm{u}}} %constructive, no comp. content
\newcommand{\tonc}{\to^{\mathrm{nc}}}
\newcommand{\falsum}{\bot}
\newcommand{\GTotal}{T} %obsolete, to be replaced by \Total

\newcommand{\nil}{\mathrm{Nil}}
%% \newcommand{\nil}{\mathrm{nil}}

\newcommand{\typeB}{\mathbf{B}}

\newcommand{\as}{\mathit{a}\!\mathit{s}}
\newcommand{\BNFdef}{\mathtt{\; ::= \;}}
\newcommand{\BNFor}{\mid}
%% \newcommand{\Eq}{\mathrm{Eq}}
\newcommand{\eqd}[2]{#1 \equiv #2}
%% \newcommand{\eqd}[2]{\mathrm{Eq}(#1,#2)}
\newcommand{\false}{\mathsf{f\kern-0.14em f}}
\newcommand{\falsityF}{\mathbf{F}}
\newcommand{\inquotes}[1]{``#1''}
\newcommand{\listappend}{\mathbin{\ast}}
\newcommand{\listrev}{\mathrm{Rv}}
\newcommand{\set}[2]{\{\,#1\mid#2\,\}}
\newcommand{\subst}[3]{#1[#2:= #3]}
\newcommand{\true}{\mathsf{t\kern-0.14em t}}

\newcommand{\mi}{Minlog}
\newcommand{\mdir}{\~{}/minlog}
\newcommand{\ob}{\to}
\newcommand{\bv}{\begin{verbatim}}

\makeindex

\newenvironment{enumeratei}{\begin{enumerate}[\upshape (i)]}
                           {\end{enumerate}}
   %% produces (i), (ii), etc,  Cross-reference with \eqref
\newenvironment{enumeratea}{\begin{enumerate}[\upshape (a)]}
                           {\end{enumerate}}
   %% produces (a), (b), etc,  Cross-reference with \eqref

\allowdisplaybreaks[4]
%% \setlength{\textheight}{19 true cm}

\hyphenation{appro-xi-ma-ted}
\hyphenation{cha-rac-te-rize}
\hyphenation{cor-res-pon-ding}
\hyphenation{de-fi-ni-tions}
\hyphenation{fa-ci-li-tate}
\hyphenation{ge-ne-ra-ted}
\hyphenation{Ma-the-ma-ti-cal}

\def\qedsymbol{{\ \vbox{\hrule\hbox{%
   \vrule height1.3ex\hskip0.8ex\vrule}\hrule}}\par}

%% minlog.mac contains the macros for all tex-files of the minlog
%% documentation in order to avoid code duplication and different
%% notations.

%% \input{minlog.mac}
\author[L. Crosilla, M. Seisenberger, H. Schwichtenberg]
{Laura Crosilla, Monika Seisenberger, Helmut Schwichtenberg}
\date{\today}

\thanks{This tutorial extends and completes a previous tutorial by L. Crosilla
distributed with version 4.0 of \mi}

\title[Minlog tutorial]{A tutorial for Minlog, version 5.0}

\begin{document}

\maketitle

\section{Introduction}
This is a tutorial for the interactive proof system \mi, Version 5.0,
developed by Helmut Schwichtenberg and members of the logic group%
\footnote{http://www.math.lmu.de/$\sim$logik/welcome.html}
at the University of Munich.

\mi\ is implemented in Scheme.
\mi's favorite dialect is Petite Chez Scheme from Cadence Research Systems,
which is freely distributed at the Internet address \url{www.scheme.com}.

The \mi\ system can be downloaded from the Internet address
\url{http://www.minlog-system.de}
%% or else http://www.math.lmu.de/$\sim$minlog/


\section{Getting started}
The purpose of this Tutorial is to give a rather basic introduction to
the \mi\ system by means of some simple examples.  For a thorough
presentation of \mi\ and the motivation behind it the reader should
consult the reference manual \cite{minlogman} and the document
\cite{tcf}.  For a more in--depth presentation of the theory
underlying \mi, the reader might find it useful also to consult the
book \cite{SchwichtenbergWainer12}.  The papers listed in
the \mi\ web page also provide a more detailed and advanced
description of specific features of the system.  In addition, the \mi\
distribution comes equipped with a directory of examples, to which the
user is referred.

In the following we shall assume tacitly that you are using an
UNIX-like operating system and that \mi\ is installed in \mdir, where
\~{} denotes as usual your home directory.  Also, C-$\langle
chr\rangle$ means: hold the CONTROL key down while typing the
character $\langle chr \rangle$, while M-$\langle chr \rangle$ means:
hold the META (or EDIT or ESC or ALT) key down while typing $\langle
chr \rangle$.

In order to use \mi, one simply needs a shell in which to run \mi\ and
also an editor in which to edit and record the commands for later
sessions.  In this tutorial we shall refer to GNU Emacs\footnote{See
  also the Appendix \ref{SS:Emacs} for some useful keyboard commands
  to start working with Emacs.}.  While working with Emacs, the ideal
would be to split the window in two parts: one containing the file in
which to store the commands, and the other with the \mi\ interactive
session taking place.  To this aim, it is recommended to use the
startup script \emph{\mdir/minlog} which takes scheme files as
(optional) arguments. For example
\begin{quote}
  \texttt{\mdir/minlog file.scm}
\end{quote}
opens a new Emacs-window which is split into two parts.  The upper
part contains the file  \emph{(Buffer file.scm)} whereas the lower part
shows the \mi\ response \emph{(Buffer *minlog*)}.

Alternatively, one can open emacs and invoke \mi\ by loading the file
\emph{minlog.el}:
\begin{quote}
  \texttt{M-x load-file <enter>}\\
  \texttt{\mdir/minlog.el}
\end{quote}
In both cases the file \emph{init.scm} is loaded. In fact, one could
also simply evaluate \texttt{(load "\mdir/init.scm")} to start \mi.

To execute a command of our file, we simply place the cursor
at the end of it (after the closed parenthesis), and type
\texttt{C-x C-e}.  In general, \texttt{C-x C-e} will
enable us to process  any command we type in
\texttt{file.scm}, one at the time.  To process a whole
series of commands, one can  highlight the region of interest and
type \texttt{C-c C-r}.  We should also mention at this
point that to undo one step, it is enough  to give the command
\texttt{(undo)}, while \texttt{(undo n)} will undo  the last $n$
steps.  Finally, we can type  \texttt{(exit)} to end a Scheme
session and  \texttt{C-x C-c} to exit Emacs.


\section{Propositional logic}

\subsection{A first example}
We shall start from a simple example in propositional logic.  Suppose
we want to prove the tautology:
\begin{equation*}
  (A \ob (B \ob C)) \ob ((A \ob B) \ob (A \ob C)).
\end{equation*}
In the following we shall make use of the convention for which
parenthesis are associated to the right (as this is also implemented
in \mi).  Therefore the formula above becomes:
\begin{equation*}
  (A \ob B \ob C) \ob (A \ob B) \ob A \ob C.
\end{equation*}
It is very important, especially at the beginning, to pay the maximum
attention to the use of parenthesis to prevent mistakes; it might be a
good strategy to rather exceed in their use in the first examples.
\mi\ will automatically delete the parenthesis which are not needed,
therefore facilitating the reading.


\subsubsection{Making a sketch of the proof}
The first task will be to make an informal sketch of the proof.  While
making the plan we should consider the following fact: \mi\ (mainly)
implements \inquotes
{Goal Driven Reasoning}, also called
\inquotes{Backward Chaining}.  That means that we start by writing the
conclusion we aim at as our goal and then, step by step, refine this
goal by applying to it appropriate logical rules.  A logical rule will
have the effect of reducing the proof of a formula, the goal, to the
proof of one or more other formulas, which will become the new goals.
If our proof is correct, then the formula will be proved when we reach
the point of having no more goals to solve.  In other words, \mi\
keeps a list of goals and updates it each time a logical rule is
applied.  The proof is completed when the list of goals is empty.

In this case the tautology we want to prove is made of a series of
implications, hence we will have to make repeated use of basic rules
for \inquotes{deconstructing} implications.  The first move will then
be to assume that the antecedent of the outmost implication is true
and try to derive the consequent from it.  That is, we assume $A \ob B
\ob C$ and want to derive:
\begin{equation*}
  (A \ob B) \ob A \ob C;
\end{equation*}
hence we set the latter as our new goal.  Then we observe that the
formula $(A \ob B) \ob A \ob C$ is an implication as well, and thus
can be treated in the same way; so we now assume both $A \ob B \ob C$
and $A \ob B$ and wish to derive $A \ob C$.  Clearly, we can make the
same step once more and obtain $A \ob B \ob C$, $A \ob B$ and $A$ as
our premises and try to derive $C$ from them.  Now we observe that in
order to prove $C$ from the assumption $A \ob B \ob C$, we need to
prove both $A$ and $B$.  Obviously $A$ is proved, as it is one of our
assumptions, and $B$ immediately follows from $A \ob B$ and $A$ by
modus ponens.


\subsubsection{Implementing the proof}
Once we have a plan for the proof, we can start implementing it in
\mi.  The initial step would then be to write the formula in \mi.  For
this purpose, we declare three predicate variables $A$, $B$ and $C$ by
writing \footnote{We could as well have introduced predicate constants
  instead of predicate variables.  In this case we would have used the
  command \texttt{add-predconst-name}, with the same syntax. }:
\begin{verbatim}
(add-pvar-name "A" "B" "C" (make-arity))
\end{verbatim}
The expression \texttt{(make-arity)} produces the empty arity for $A$,
$B$ and $C$ (see \cite{minlogman} for a description of
\texttt{make-arity}).  \mi\ will then write:
\begin{verbatim}
ok, predicate variable A: (arity) added
ok, predicate variable B: (arity) added
ok, predicate variable C: (arity) added
\end{verbatim}

We now want to prove the above formula with \mi; we thus need to set
it as our goal.


\subsubsection{Setting the goal}
To set a goal, we can use the command \texttt{set-goal} followed by
the formula.  In the present case:
\begin{verbatim}
(set-goal "(A -> B -> C) -> (A -> B) -> A -> C")
\end{verbatim}
Alternatively, we can first give a name to the formula we wish to
prove and then use this name to set the goal.  In the present case,
let's call \texttt{distr} (for distributivity of implication) the
formula to be proved:
\begin{verbatim}
(define distr (pf "(A -> B -> C) -> (A -> B) -> A -> C"))
\end{verbatim}

The \texttt{define} command has the effect of defining a new variable,
in this case \texttt{distr}, and attaching to it the Scheme term which
is produced by the function \texttt{pf} applied to the formula we
enter.  In fact, the function {\texttt pf}, short for \inquotes{parse
  formula}, takes a string as argument and returns a Scheme term.
This Scheme term is the \emph{internal form} in \mi\ of our formula,
and \texttt{distr} is a name referring to it.  By typing
\texttt{distr}, one can see the value of this variable.  The strategy
of naming a formula might turn out to be particularly useful in case
of very long goals.

To set \texttt{distr} as our goal we type:
\begin{verbatim}
(set-goal distr)
\end{verbatim}
Typically, \mi\ will number the goals occurring in the proof, and will
display the top goal as number 1, preceded by a question mark.  \mi\
will print:
\begin{verbatim}
?_1: (A -> B -> C) -> (A -> B) -> A -> C
\end{verbatim}


\subsubsection{The proof}
According to our sketch, the first step in proving the tautology was
to assume the antecedent of the implication and turn the consequent
into our new goal.  This is simply done by writing:
\begin{verbatim}
(assume 1)
\end{verbatim}

Here the number \texttt{1} is introduced by us to identify and name
the hypothesis. \mi\ will thus denote this hypothesis by \texttt{1}:
\begin{verbatim}
ok, we now have the new goal
?_2: (A -> B) -> A -> C from
  1:A -> B -> C
\end{verbatim}

We repeat the assume command to decompose the implication in the
second goal and obtain a new goal:
\begin{verbatim}
(assume 2)
\end{verbatim}

\begin{verbatim}
ok, we now have the new goal
?_3: A -> C from
  1:A -> B -> C
  2:A -> B
\end{verbatim}

And we decompose the new goal once more:
\begin{verbatim}
(assume 3)
\end{verbatim}

\begin{verbatim}
ok, we now have the new goal
?_4: C from
  1:A -> B -> C
  2:A -> B
  3:A
\end{verbatim}

We now need to start using our assumptions.  As already mentioned, in
order to prove $C$ it is enough to prove both $A$ and $B$, by
assumption \texttt{1}.  Therefore we write: \texttt{(use 1)}.
This has the effect of splitting the goal in two distinct subgoals
(note how the subgoals are numbered):
\begin{verbatim}
(use 1)
ok, ?_4 can be obtained from
?_6: B from
  1:A -> B -> C
  2:A -> B
  3:A
?_5: A from
  1:A -> B -> C
  2:A -> B
  3:A
\end{verbatim}

Then we write:
\begin{verbatim} (use 3)
ok, ?_5 is proved.  The active goal now is
?_6: B from
  1:A -> B -> C
  2:A -> B
  3:A
\end{verbatim}

And conclude the proof by:
\begin{verbatim}
(use 2)
ok, ?_6 can be obtained from
?_7: A from
  1:A -> B -> C
  2:A -> B
  3:A
>
(use 3)
ok, ?_7 is proved.  Proof finished.
\end{verbatim}

To see a record of the complete proof, simply type
\texttt{(display-proof)}.  Other useful commands are
\texttt{(proof-to-expr)} and the particularly useful
\texttt{(proof-to-expr-with-formulas)}.  See the manual for a
description of the various display commands available in \mi.

We observe that the first three \texttt{assume} commands could be
replaced by a single one, i.e., \texttt{(assume 1 2 3)}.  Also, in
alternative to the last two \texttt{use} commands, we could have given
only one command: \texttt{(use-with 2 3)}, which amounts to applying a
cut to the premises 2 and 3.  A final remark: in case of rather
complex proofs, it may be more convenient to use names to denote
specific hypothesis, instead of making use of bare numbers.  To do so,
one can simply use the \texttt{assume} command, followed by the name
of the assumption in double quotes.

Before starting to read the next section it is advisable to consult
the reference manual \cite{minlogman} for a compendium of the commands
utilized in this example.  It is worth noticing that in general these
commands have a wider applicability than their usage as here
presented.


\subsection{A second example: conjunction}
The next example is a simple tautology made of conjunctions and an
implication.  We want to prove\footnote{Recall that $\land$ binds
  stronger than $\ob$.}:
\begin{equation*}
  A \land B \ob B \land A.
\end{equation*}
In this case, we shall simply record the code of our \mi\ proof,
asking the reader to check \mi's reply at each step.  We start as
usual with declaring the variables $A$ and $B$ and setting the goal.
Note that if one uses the same file for a number of examples, there is
no need to re-declare the same predicate variables each time.  Hence
here and in the example file available with the distribution repeated
declarations are commented by prefixing a \inquotes{;; }.
\begin{verbatim}
;; (add-pvar-name "A" "B" (make-arity))
(set-goal "A & B -> B & A")
\end{verbatim}

We than notice that the main connective is an implication, and thus
call the command \texttt{assume}:
\begin{verbatim}
(assume 1)
\end{verbatim}

Next we need a command which operates on a conjunction and splits it
into its two components. We can simply write:
\begin{verbatim}
(split)
\end{verbatim}

Finally, we impart the command \texttt{use} which is utilized to
obtain the left (respectively the right) conjunct from an assumption
which is a conjunction and \inquotes{\texttt{use}} it to derive the
goal.
\begin{verbatim}
(use 1)
(use 1)
\end{verbatim}
This completes the proof.


\subsection{Exercises}
The reader is encouraged to try and prove other tautologies.  For
example the following:
\begin{enumerate}
\item $A \to B \to A$
\item $(A \to B \to C) \to B \to A \to C$
\item $(A \to B) \to (B \to C) \to A \to C$
\item $(A \to B \to C) \to A \land B \to C$
\end{enumerate}


\subsection{Classical logic}
To conclude this section on propositional logic, we give a short
example of a tautology which uses classical logic.

\mi\ implements minimal logic.  However, it is possible to use \mi\ to
prove a proposition which holds in an extension of minimal logic, like
intuitionistic or classical logic.  This is achieved by adding
specific principles which are characteristics of each kind of logic,
like the \inquotes{ex falso quodlibet} or a version of the
\inquotes{tertium non datur}.  We recall that intuitionistic logic is
obtained from minimal logic by adding a principle which enables us to
derive any formula from falsity (\inquotes{ex falso quodlibet} is
Latin for \inquotes{anything follows from falsity}).  This principle
is usually stated as: $\falsum \ob A$, for arbitrary $A$, where
$\falsum$ denotes falsity.  Classical logic may be obtained by adding
to intuitionistic logic the law of \inquotes{tertium non datur} (Latin
for \inquotes{there is no third option}).  This is usually stated as
$A \lor \neg A$, for any $A$.  However, classical logic may also be
obtained by adding to minimal logic a consequence of \inquotes{tertium
  non datur}, known as \inquotes{stability}, asserting that $\neg \neg
A \to A$, for any $A$.  Negation is represented in \mi\ as follows:
$\neg A $ is $A \to \falsum$; consequently stability is written as:
\begin{equation*}((A \ob \falsum) \ob \falsum) \ob A,\end{equation*}
for each $A$.

We can add \inquotes{external} principles to \mi\ by introducing
so--called \emph{\inquotes{global assumptions}}.  Roughly speaking, a
global assumption is a proposition whose proof does not concern us at
the moment; hence it can also be an assumption with no proof.  Some
global assumptions are already set by default, like \texttt{EfqLog}
(which is \mi's name for \inquotes{ex falso quodlibet}) and
\texttt{StabLog} (which is \mi's name for the law of stability).  In
order to check which global assumptions we have at our disposal we
type: \texttt{(display-global-assumptions)}.  To check a particular
global assumption (or theorem) whose name we already know, we write
\texttt{pp}\index{pp@\texttt{pp}} (for pretty-print) followed by the
name of the assumption (or theorem) we want to check, e.g.:
\texttt{(pp "StabLog")}.  Of course we can also introduce our own
global assumptions and remove them at any time (see the reference
manual for the relevant commands).

In the following we wish to prove the tautology:
\begin{equation*}
  ((A \ob B) \ob A) \ob A,
\end{equation*}
which is known as \emph{Peirce formula}.  Also for this example, we
will assume that the reader has prepared her sketch of the proof, and
we will only give an intuitive idea of the proof, preferring to rather
concentrate on the \mi\ interaction, which will be given in its
complete form.

As in the previous examples, we observe first of all that the goal is
an implication, hence we will assume its antecedent, $(A \ob B) \ob
A$, and try to prove its consequent, $A$.  Now classical logic comes
into play: in order to prove $A$ we will assume that its negation
holds and try to obtain a contradiction from it.  This will be
achieved by use of Stability.  We further note that in order to make
the argument work, we will need at some stage to resort also to
\inquotes{ex falso quodlibet}.

We start by setting the goal and assuming the antecedent of the
implication:
\begin{verbatim}
;; (add-pvar-name "A" "B" (make-arity))
(set-goal "((A -> B) -> A) -> A")
(assume 1)
\end{verbatim}

We obtain:
\begin{verbatim}
ok, we now have the new goal
?_2: A from
  1:(A -> B) -> A
\end{verbatim}

We now apply Stability, \texttt{StabLog}, so that the goal $A$ will
be replaced by its double negation: $(A \ob \falsum) \ob \falsum$.
Note that $\falsum$ is called \texttt{bot} in \mi.

\begin{verbatim}
(use "StabLog")
ok, ?_2 can be obtained from
?_3: (A -> bot) -> bot from
  1:(A -> B) -> A
\end{verbatim}

Since this is an implication, we let:
\begin{verbatim}
(assume 2)
ok, we now have the new goal
?_4: bot from
  1:(A -> B) -> A
  2:A -> bot
\end{verbatim}

We then use hypothesis 2 to replace the goal $\falsum$ by $A$.
\begin{verbatim}
(use 2)
ok, ?_4 can be obtained from
?_5: A from
  1:(A -> B) -> A
  2:A -> bot
\end{verbatim}

Also $A$ can be replaced by $A \ob B$ by use of hypothesis 1.
Subsequently, we can assume the antecedent of the new goal, $A$, and
call it hypothesis 3:
\begin{verbatim}
(use 1)
ok, ?_5 can be obtained from
?_6: A -> B from
  1:(A -> B) -> A
  2:A -> bot
>
(assume 3)
ok, we now have the new goal
?_7: B from
  1:(A -> B) -> A
  2:A -> bot
  3:A
\end{verbatim}

Now we can make use of the principle of \inquotes{ex falso quodlibet}:
if we want to prove $B$, we can instead prove falsum, since from
falsum anything follows, in particular $B$.  Our goal can be updated
to $\falsum$ by the following instance of \texttt{use}:
\begin{verbatim}
(use "EfqLog")
ok, ?_7 can be obtained from:
?_8: bot from
  1:(A -> B) -> A
  2:A -> bot
  3:A
\end{verbatim}

The next two steps are obvious.
\begin{verbatim}
(use 2)
ok, ?_8 can be obtained from
?_9: A from
  1:(A -> B) -> A
  2:A -> bot
  3:A
>
(use 3)
ok, ?_9 is proved.  Proof finished.
\end{verbatim}


\subsection{Exercises}
To familiarize yourself with negation, prove the following
propositions.
\begin{enumerate}
  \item $(A \to B) \to \neg B \to \neg A$,
  \item $\neg (A \to B) \to \neg B$,
  \item $\neg \neg (A \to B) \to \neg \neg A \to \neg \neg B$.
\end{enumerate}


\subsection{Note on disjunction}
In \mi\ we don't have a primitive logical constant for $\lor$.  This
is due to the fact that when wishing to prove the disjunction $A \lor
B$ we can often prove
\begin{verbatim}
(A -> Pvar) -> (B -> Pvar) -> Pvar
\end{verbatim}
instead\footnote{Note, however, that there is also an (inductively
  defined) \inquotes{or}, which is displayed as \texttt{ord}, with
  rules:

\texttt{(pp "InlOrD")}

\texttt{Pvar1 -> Pvar1 ord Pvar2}

\texttt{(pp "InrOrD")}

\texttt{Pvar2 -> Pvar1 ord Pvar2}
}.


\section{Predicate logic}

\subsection{A first example with quantifiers}
\label{Quantifiers}
We now exemplify how to prove a statement in predicate logic.
We want to prove:
\begin{equation*}
  \forall_n (P n \to Q n) \to \forall_n P n \to \forall_n Q n.
\end{equation*}
Here we assume that the predicates $P$ and $Q$ take natural numbers as
arguments.  Therefore, we first of all load a file, already available
within the distribution, which introduces the algebra of natural
numbers, including some operations on them, like for example addition.
The reader is advised to have a look at this file\footnote{See also
  section \ref{NaturalNumbers}.} by typing:

\texttt{(libload "nat.scm")}

Note that this command will produce the display of the whole
file nat.scm, evaluated.
If we wish to load the file \inquotes{silently}, then we can
precede the libload command by the following line:
\begin{verbatim}
(set! COMMENT-FLAG #f)
\end{verbatim}
This will have the effect of hiding the output.  To revert to full
display (which is needed to proceed with the proof) one types:
\begin{verbatim}
(set! COMMENT-FLAG #t)
\end{verbatim}

As we load \texttt{nat.scm}, we can make use of all the conventions
which are there stipulated; in particular, we can take $n,m,k$ to be
variables for natural numbers (i.e. of type
\texttt{nat})\footnote{Note that \mi\ automatically infers the type of
  the variables $x_0, x_1, \dots$ once the variable $x$ has been
  declared. In this case, it infers for example that $n_0, n_1, \dots$
  are also natural numbers.}.  The next task is to introduce two new
predicate variables $P$ and $Q$ which take natural numbers as
arguments:
\begin{verbatim}
(add-pvar-name "P" "Q" (make-arity (py "nat")))
> ok, predicate variable P: (arity nat) added
ok, predicate variable Q: (arity nat) added
\end{verbatim}

We then set the goal:
\begin{verbatim}
(set-goal "all n(P n -> Q n) -> all n P n -> all n Q n")
> ?_1: all n(P n -> Q n) -> all n P n -> all n Q n
\end{verbatim}

As usual, we first have to \inquotes{deconstruct} the implications:
\begin{verbatim}
(assume 1 2)
> ok, we now have the new goal
?_2: all n Q n from
  1:all n(P n -> Q n)
  2:all n P n
\end{verbatim}

Then we need to take care of the universal quantifier in the goal:
\begin{verbatim}
(assume "n")
> ok, we now have the new goal
?_3: Q n from
  1:all n(P n -> Q n)
  2:all n P n
  n
\end{verbatim}

Note that we could have also used only one command to perform all
these actions:
\begin{verbatim}
(assume 1 2 "n")
\end{verbatim}

We finally have to \inquotes{use} our hypothesis to conclude the proof:
\begin{verbatim}
(use 1)
(use 2)
> ok, ?_3 can be obtained from
?_4: P n from
  1:all n(P n -> Q n)
  2:all n P n
  n
> ok, ?_4 is proved.  Proof finished.
\end{verbatim}


\subsection{Another example}
We now wish to prove the following:
\begin{equation*}
  \forall_n (P n \to Q n) \to \exists_n P n \to \exists_n Q n.
\end{equation*}
We start by setting the goal and eliminating the two implications:
\begin{verbatim}
(set-goal "all n(P n -> Q n) -> ex n P n -> ex n Q n")
(assume 1 2)
?_1: all n(P n -> Q n) -> ex n P n -> ex n Q n
> ok, we now have the new goal
?_2: ex n Q n from
  1:all n(P n -> Q n)
  2:ex n P n
\end{verbatim}

Next we want to use the second assumption, now by applying
\inquotes{forward reasoning}. Thus we assume there is a witness, say
$n_0$, for this existential formula and call the resulting hypothesis
$P_{n_0}$.  To this aim we make use of a command called
\texttt{by-assume}:
\begin{verbatim}
(by-assume 2 "n0" "P_n0")
> ok, we now have the new goal
?_5: ex n Q n from
  1:all n(P n -> Q n)
  n0  P_n0:P n0
\end{verbatim}

Alternatively, from the existential formula in hypothesis 2, we can
extract a witness, say $n_0$, by applying an existential elimination.
\begin{verbatim}
(ex-elim 2)
(assume "n0" "P_n0")
> ok, ?_2 can be obtained from
?_3: all n(P n -> ex n0 Q n0) from
  1:all n(P n -> Q n)
  2:ex n P n
> ok, we now have the new goal
?_4: ex n Q n from
  1:all n(P n -> Q n)
  2:ex n P n
  n0  P_n0:P n0
\end{verbatim}
As we have already used assumption 2, we may wish to drop it by
writing:
\begin{verbatim}
(drop 2)
\end{verbatim}

We conclude the proof by providing a witness, the term $n_0$, for our
existential goal formula.  This will be done by calling the command
\texttt{ex-intro} with argument \texttt{(pt "n0")}, where \texttt{pt}
stands for \inquotes{parse term}.
\begin{verbatim}
(ex-intro (pt "n0"))
> ok, ?_5 can be obtained from
?_6: Q n0 from
  1:all n(P n -> Q n)
  n0  P_n0:P n0
\end{verbatim}

Finally, we first instantiate the universal quantifier in assumption 1
to $n_0$ and then perform a cut with the third assumption.
\begin{verbatim}
(use-with 1 (pt "n0") "P_n0")
>  ok, ?_5 is proved.  Proof finished.
\end{verbatim}


\subsection{An example with relations}
In the next example we wish to prove that every total relation which
is symmetric and transitive is reflexive.  For simplicity we shall
work also in this case with the algebra of the natural numbers.  Our
aim is to prove the following statement:
\begin{align*}
        &       \forall_{n,m}\, (R n m \ob R m n)
                \,\land\, \forall_{n, m, k}\,
                (R n m \;\land\; R m k \ob R n k)\\
        &       \ob \forall_n\, (\exists_m\, R n m \ob R n n),
\end{align*}
\noindent
where $n$, $m$, $k$ vary on natural numbers, while $R$ is a binary
predicate on natural numbers.

Before attacking our formula, we observe that in general conjunctions
are quite complex to deal with, as they normally imply the branching
of a proof in two subproofs.  Thus we might wish to first find a
formula which is equivalent to the one above and \inquotes{simpler} to
prove.  We note that we can equivalently express our goal by a formula
in which the conjunctions have been replaced by implications. Also, we
can express the conclusion with a prenex universal quantifier instead
of an existential one.  That is, we can instead prove the following
equivalent formula:
\begin{align*}
        &       \forall_{n, m}\, (R n m \ob R m n)
                \ob \forall_{n, m, k}\,
                (R n m \ob R m k \ob R n k)\\
        &       \ob \forall_{n, m}\, (R n m \ob R n n).
\end{align*}
We observe that the strategy of first simplifying the goal may in some
cases allow one to considerably reduce the amount of time needed to
prove a statement.  For completeness and for a comparison, we shall
also record a proof of the original goal at the end of this section.

We now start by introducing the constant $R$. We also want to
facilitate our work a bit further and introduce names for our two
assumptions.  In the following we shall use the function \texttt{py}
(for \inquotes{parse type}), which is the analogous for types of the
function parse formula that we encountered in the first example.
\begin{verbatim}
(add-pvar-name "R" (make-arity (py "nat") (py "nat")))
(define Sym (pf "all n,m(R n m  -> R m n)"))
(define Trans (pf "all n,m,k(R n m -> R m k -> R n k)"))
\end{verbatim}

We now state the goal:
\begin{verbatim}
(set-goal (mk-imp Sym Trans (pf "all n,m(R n m -> R n n)")))
?_1: all n,m(R n m -> R m n)
       -> all n,m,k(R n m -> R m k -> R n k)
       -> all n,m(R n m -> R n n)
\end{verbatim}

Note that also in this case, we could have directly written the two
formulas as antecedents of the implication, avoiding the detour
through a \texttt{define} command.  In case of more complex formulas,
however, or when we need to use the same formulas for various proofs
through one session, the strategy of introducing names for assumptions
can be quite useful.

We now observe that the goal is an implication, so that the first step
is to write \texttt{(assume "Sym" "Trans")}.  We now obtain a
universally quantified formula and hence need to proceed to eliminate
the quantifiers. This can be accomplished by another \texttt{assume}
command in which we specify two natural numbers, say $n$ and $m$.  So
we write \texttt{(assume "n" "m")}.  This produces an implication
which again needs to be eliminated by another \texttt{assume} command,
say \texttt{(assume 3)}.  Quite conveniently we can put all these
commands together by simply writing:
\begin{verbatim}
(assume "Sym" "Trans" "n" "m" 3)
ok, we now have the new goal
?_2: R n n from
  Sym:all n,m(R n m -> R m n)
  Trans:all n,m,k(R n m -> R m k -> R n k)
  n  m  3:R n m
\end{verbatim}

The next move is to make use of our assumptions.  It is clear that if
we take $k$ to be $n$ in \texttt{Trans}, then the goal can be obtained
by an instance of \texttt{Sym}, and the proof is easily completed.
We here utilize \texttt{use} by additionally providing a term,
\texttt{"m"}, which instantiates the only variable which can not be
automatically inferred by unification\footnote{See \cite{tcf} for an
  introduction to unification.}.
\begin{verbatim}
(use "Trans" (pt "m"))
?_4: R m n from
  Sym:all n,m (R n m -> R m n)
  Trans:all n,m,k(R n m -> R m k -> R n k)
  n  m  3:R n m
?_3: R n m from
  Sym:all n,m(R n m -> R m n)
  Trans:all n,m,k(R n m -> R m k -> R n k)
  n  m  3:R n m
\end{verbatim}

The \texttt{use} command has the effect of replacing the current goal
with two new goals.  These are obtained from \texttt{Trans} by
instantiating the quantifiers with \texttt{n}, \texttt{m} and
\texttt{n} (the two \texttt{n} being inferred by unification) and then
by replacing the goal with the antecedents of the resulting instance
of \texttt{Trans}.  We can now write:
\begin{verbatim}
(use 3)
> ok, ?_3 is proved.  The active goal now is
?_4: R m n from
  Sym:all n,m(R n m -> R m n)
  Trans:all n,m,k(R n m -> R m k -> R n k)
  n  m  3:R n m
\end{verbatim}

We finally employ \texttt{Sym} and another \texttt{use}:
\begin{verbatim}
(use "Sym")
ok, ?_4 can be obtained from
?_5: R n m from
  Sym:all n,m(R n m -> R m n)
  Trans:all n,m,k(R n m -> R m k -> R n k)
  n  m  3:R n m
>
(use 3)
ok, ?_5 is proved.  Proof finished.
\end{verbatim}


\subsection{The same example again}
We here present a \mi\ proof of the original goal in the previous
example, as it allows us to exemplify the use of some new commands.
We shall leave the proof uncommented and make a few remarks at the
end. The reader will have to examine the proof and check \mi's
interaction.
\begin{verbatim}
;; (libload "nat.scm")
;; (add-pvar-name "R" (make-arity (py "nat") (py "nat")))
(set-goal "all n,m(R n m  -> R m n)
                 & all n,m,k(R n m & R m k -> R n k)
                 -> all n(ex m R n m -> R n n)")
(assume 1)
(inst-with 1  'left)
(inst-with 1 'right)
(drop 1)
(name-hyp 2 "Sym")
(name-hyp 3 "Trans")
(assume "n" 4)
(ex-elim 4)
(assume "m" 5)
(cut "R m n")
(assume 6)
(use-with "Trans" (pt "n") (pt "m") (pt "n") "?")
(drop "Sym" "Trans" 4)
(split)
(use 5)
(use 6)
(use-with "Sym" (pt "n") (pt "m") 5)
\end{verbatim}

The \texttt{use-with} command is similar to the \texttt{use} command,
but when applied to a universal quantifier it requires to explicitly
specify the terms one wants to instantiate.  In the second occurrence
of \texttt{use-with}, \mi\ will instantiate as specified the universal
quantifiers in the second premise and then use hypothesis 5 to prove
the goal.

The command \texttt{inst-with} is analogous to \texttt{use-with}, but
operates in forward reasoning; hence it allows one to simplify the
hypothesis, instead of the conclusion. In this case,
\texttt{(inst-with 1 'left)} has the effect of producing the left
component of the conjunction which constitutes the first hypothesis.
Similarly for the right component.

As to \texttt{cut}, this command enables one to introduce new goals:
\texttt{(cut A)} has the effect of replacing goal $B$ by two new
goals, $A \ob B$ and $A$.

In the proof above we have also made use of the commands \texttt{drop}
and \texttt{name-hyp}.  We have already seen the first command, which
allows one to remove one or more hypothesis from the present context,
to make the proof more readable.  In fact, it simply replaces the
current goal with another goal in which the hypothesis
\inquotes{dropped} are not displayed anymore (but they are not removed
in general, as should be clear from the example above).  The second
command has similar \inquotes{cosmetic} purposes, and allows one to
rename a specific hypothesis and hence to work with names given by the
user instead of numbers produced by default.  Both these commands
result especially useful in the case of long and intricate proofs.


\subsection{Exercises}
Prove the following goals:
\begin{enumerate}
\item $\forall_{m,n} R m n \to \forall_{n,m} R m n$
\item $\forall_{m, n} R m n \to \forall_n R n n$
\item $\exists_m \forall_n R m n  \to \forall_n \exists_m R m n$
\end{enumerate}


\subsection{Advanced exercises}
Now two examples which involve a function.  First declare a new
function variable (where \texttt{av} stands for \inquotes{add
  variable}):
\begin{verbatim}
(av "f" (py "nat=>nat"))
(set-goal "all f(all n(P(f n) -> Q n)
-> all n P n -> all n Q n)")
(set-goal "all f(all n(P n -> Q (f n))
-> ex n P n -> ex n Q n)")
\end{verbatim}

And finally:
\begin{verbatim}
;; (add-pvar-name "Q" (make-arity (py "nat")))
;; (add-pvar-name "A" (make-arity))
(set-goal "all n(Q n -> A) -> (ex n Q n -> A)")
(set-goal "ex n(Q n -> A) -> all n Q n -> A")
\end{verbatim}


\subsection{Another example with classical logic}
We conclude this section on predicate logic with the proofs of two
formulas which hold in classical logic.  First of all, we prove the
inverse of the formula in the last exercise, now generalised to an
arbitrary type.  Then we use this formula (conveniently stored as a
Lemma) to prove another formula which is usually known as the
\inquotes{Drinker} formula.  So we start by proving:
\begin{equation*}
  (\forall_x Q x \to A) \to \excl_x (Q x \to A).
\end{equation*}
Here $Q$ is a unary predicate which ranges on an arbitrary type, say
$\alpha$.  In addition, the existential quantifier, $\excl$, is here a
\textbf{classical} existential quantifier, to be distinguished from
the existential quantifier we encountered in the previous example. A
classical quantifier $\excl_x$ is nothing more than an abbreviation
for $\lnot\, \forall_x\, \lnot$.  Note that \mi\ implements both
quantifiers, with the appropriate corresponding rules.

We start by \inquotes{removing} the constant $Q$ from example
\ref{Quantifiers}, so that we can re-introduce it as a fresh constant
which ranges on $\alpha$ rather than on the natural numbers. We also
introduce two new variables, $x$ and $y$, of type $\alpha$. Finally,
we set the goal.
\begin{verbatim}
(remove-pvar-name "Q")
(add-pvar-name "Q" (make-arity (py "alpha")))
(av "x" (py "alpha"))
(set-goal "(all x Q x -> A) -> excl x(Q x -> A)")
\end{verbatim}

We start by \inquotes{deconstructing} the two implications.  Then we
can instantiate the universal quantifier in assumption 2 to a
canonical inhabitant of the type $\alpha$.
\begin{verbatim}
(assume 1 2)
(use 2 (pt "(Inhab alpha)"))
\end{verbatim}

Subsequently, we proceed by eliminating the implication in the goal,
using assumption 1 and instantiating the resulting universal
quantifier by $x$.
\begin{verbatim}
(assume 3)
(use 1)
(assume "x")
\end{verbatim}

Now it's time to call in classical logic.  The remaining steps should
be self--explanatory.  Note in particular the use of \texttt{EfqLog}
and the command \texttt{save} at the end of the proof which enables us
to save the proof and call it \inquotes{Lemma}.
\begin{verbatim}
(use "StabLog")
\end{verbatim}

\begin{verbatim}
(assume 4)
(use 2 (pt "x"))
(assume 5)
(use 1)
(assume "x1")
\end{verbatim}

\begin{verbatim}
(use "EfqLog")
(use-with 4 5)
(save "Lemma")
\end{verbatim}

We now wish to prove the following:
\begin{equation*}
  \excl_x\, (Q x \ob \forall_y\, Q y),
\end{equation*}
again with $Q$ a unary predicate ranging on $\alpha$.

The above formula is known as the \inquotes{drinker} formula, as it
says something like: \inquotes{in a bar, there is a person such that
  if that person drinks then everybody drinks}.  To prove the
\inquotes{drinker}, we observe that if we substitute the predicate
variable $A$ by $\forall_x Q x$ in the formula just proved, then we
obtain the drinker formula.  The substitution can be achieved by the
following command.
\begin{verbatim}
(set-goal "excl x(Q x -> all x Q x)")
(use-with "Lemma"
          (make-cterm (pv "x") (pf "Q x"))
          (make-cterm (pf "all x Q x"))
          "?")
\end{verbatim}

Here \texttt{pv} stands for \inquotes{parse variable}.  In addition,
\texttt{make-cterm} produces a \emph{\inquotes{comprehension term}}
consisting of a list of variables and the formula we wish to
substitute.  For example, if we wish to replace a predicate variable
$P$ with arity $x_1, \dots, x_n$ by a formula $F (y_1, \dots, y_n)$ we
need to give a comprehension term consisting of a list of variables
$y_1, \dots, y_n$ and the formula $F$ (with free variables $y_1,
\dots, y_n$, plus possibly other variables, bound or free). That is,
we write: \texttt{(make-cterm (pv "y1") ...(pv "yn") <formula F> )}.
Note that the list of variables can also be empty, as in the second
application of \texttt{make-cterm} above.  We leave the rest of the
proof as an exercise for the reader.


\subsection{Equality reasoning}
We now wish to prove that for any function $f$ taking natural numbers
to natural numbers, for any natural number $n$, the following holds:
\begin{equation*}
  f n = n \to  f (f n) = n.
\end{equation*}
First of all we recall the nat library and introduce $f$.  Then we set
the goal and start by a familiar \texttt{assume} command.
\begin{verbatim}
;; (libload "nat.scm")
(av "f" (py "nat=>nat"))

(set-goal "all f,n(f n=n ->  f(f n)=n)")
(assume "f" "n" 1)
\end{verbatim}

Next we can use the command \texttt{simp} which is an essential tool
in \mi 's equality reasoning. This command has the effect of
\emph{simplifying} a proof which involves equal terms by performing an
appropriate substitution in the goal.  We conclude with a \texttt{use}
command.
\begin{verbatim}
(simp 1)
(use 1)
\end{verbatim}

Suppose now we wish to replace the right hand side by the left hand
side in the equation above:
\begin{verbatim}
(set-goal "all f,n(n=f n -> n=f(f n))")
\end{verbatim}

This can be proved by the following commands:
\begin{verbatim}
(assume "f" "n" 1)
(simp "<-" 1)
(use 1)
\end{verbatim}


\section{Automatic proof search}
Minlog allows for automatic proof search.  There are two distinct
facilities for performing an automatic search in Minlog.  The first is
given by the command \texttt{(prop)} and exemplifies
Hudelmaier-Dyckhoff's search for the case of minimal propositional
logic (see e.g.\ \cite{Hudelmaier89}, \cite{Dyckhoff92}).  The second
is given by the command \texttt{(search)} and allows to automatically
find proofs also for some quantified formulas.

\subsection{Search in propositional logic}
When we give the command \texttt{prop}, \mi\ will first look for a
proof in propositional minimal logic.  If it fails to find a proof for
the given proposition, it will try with intuitionistic logic, by
adding appropriate instances of \inquotes{ex falso quodlibet}.  If
this search also gives no positive answer, it will try to find a proof
in classical logic, by adding appropriate instances of Stability.

To apply this search algorithm, one simply needs to type
\texttt{(prop)}. One could do so after stating the goal or at any
point in a proof from which one believes that (minimal) propositional
logic should suffice.  If Minlog finds a proof, one can then
display it by means of any of the display commands available for
proofs; for example by writing \texttt{dnp} (which is a shortcut for
\texttt{display-normalized-proof}).

The reader is encouraged to try \texttt{prop} on
the following tautologies:
\begin{enumerate}
\item $(A \to B \to C) \to (A \to B) \to A \to C$
\item $((A \to B) \to A) \to A$
\end{enumerate}

Further test examples can be found in the section on
propositional logic in this tutorial.


\subsection{Search in predicate logic}
The command \texttt{search} embodies a search algorithm based on
\cite{Miller91b} and ideas of U.~Berger (see the Minlog reference
manual and \cite{Schwichtenberg04} for details on the algorithm and
for some differences with Miller's original algorithm).  The
\texttt{search} command enables us to automatically find a proof for a
wider class of formulas compared with \texttt{prop}, since it also
works for some formulae with quantifiers (see the reference manual for
a detailed description of the class of formulae dealt with by
\texttt{search}).  Note, however, that \texttt{search} only operates a
search in \emph{minimal logic}.  If one wishes to apply this command
to a classical formula like \inquotes{Peirce's law}, one could for
example add the appropriate instances of \inquotes{ex falso quodlibet}
and of \inquotes{Stability} as antecedents of the goal.  In case of
more complex proofs, in which one can not easily modify the actual
goal, an alternative would be to avail oneself of a more complete use
of the \texttt{search} command which allows us to specify some global
assumptions, theorems or even hypotheses from the given context which
one would like to use in the proof.  Since the search space in the
case of quantified formulas can become really vast, this possibility
of declaring specific assumptions to be used in the proof can be very
useful, especially if we also state the maximum number of
multiplicities we allow for each assumption (i.e., the maximum number
of times each assumption can be used in the proof).  One can also use
this same device to exclude the use of a specific assumption in the
proof, simply by letting its multiplicity to be $0$.

To use the plain version of \texttt{search}, one simply writes
\texttt{(search)}.  See the reference manual for the precise syntax of
the command \texttt{search} when other assumptions are invoked with
the respective multiplicities.

The reader is encouraged to use \texttt{search} to prove the
following: $\forall_x (P x \to Q x) \to \exists_x P x \to \exists_x Q
x$.


\subsubsection{A more complex example with \texttt{search}}
We here wish to introduce a more complex example for the use of
\texttt{search}.  We apply the algorithm to the following problem: if
$f$ is a continuous function then $f$ composed with itself is also a
continuous function.  We suggest to solve the problem as follows.
\begin{verbatim}
;; (add-var-name "x" "y" (py "alpha"))
(add-tvar-name "beta")
(add-var-name "u" "v" "w" (py "beta"))
(add-program-constant "In" (py "alpha=>beta=>boole"))
(add-infix-display-string "In" "elem" 'rel-op)
(add-var-name "f" (py "alpha=>alpha"))

(set-goal "all f(
 all x,v(f x elem v -> 
  excl u(x elem u & all y(y elem u -> f y elem v))) ->
 all x,w(f(f x)elem w -> 
  excl u(x elem u & all y(y elem u -> f(f y)elem w))))")
(search)
\end{verbatim}
Note that one can switch on a verbose form of search by letting:
\texttt{(set! VERBOSE-SEARCH \#t)} before calling \texttt{search}. In
this way one can see the single steps performed by the search
algorithm and detect possible difficulties in finding a proof.

Also, \texttt{add-infix-display-string} allows us to define a token
with infix notation for the program constant\footnote{Similarly there
  are commands for prefix and postfix use, for example:
  \texttt{add-prefix-display-string}.}.

\section{Datatypes and inductively defined predicates}
\label{Datatypes}
At this point we prefer to slightly change our style how goals appear
on the screen.  The reason is that from now on our proofs become
somewhat more complex, and then the new style is more readable.
\begin{itemize}
\item For somewhat longer contexts it is clearer to list them first
  and the goal below it, separated by a line.  The command to switch
  to this display style is
\begin{verbatim}
(set! COQ-GOAL-DISPLAY #t)
\end{verbatim}
\item We avoid using numbers to identify hypotheses and rather use
  strings indicating what kind of a hypothesis we have.  For example
  \inquotes{IHn} tells us that this is the hypothesis of an induction
  over $n$.
\end{itemize}

\subsection{The natural numbers}
\label{NaturalNumbers}
The standard example of a datatype is that of the natural numbers.  We
have already seen that the natural numbers are implemented in \mi\ as
an algebra, and that the distribution comes equipped with a file,
called \texttt{nat.scm}, which introduces this algebra.  The algebra's
constructors are $0$ and $Succ$ (zero and successor).  To display
these constructors, we simply write:
\begin{verbatim}
(display-alg "nat")
\end{verbatim}
We obtain \mi's reply:
\begin{verbatim}
> nat
        Zero:        nat
        Succ:        nat=>nat
\end{verbatim}

Note also that for convenience \mi\ allows us to write \texttt{0, 1, 2,
  3, ...} instead of \texttt{Zero, (Succ Zero), Succ(Succ Zero), ...}.

Algebras usually come equipped with some functions, which are called
\textbf{program constants} in \mi. For example, in the case of the
natural numbers, one has the program-constants \texttt{NatPlus} and
\texttt{NatTimes}, for addition and multiplication, respectively.
These are displayed as \texttt{+} and \texttt{*}.  The behaviour of
program constants is specified by means of appropriate term rewriting
rules which in \mi\ are called \textbf{computation rules} and
\textbf{rewrite rules} \footnote{The idea is that a computation rule
  can be understood as a description of a computation in a suitable
  \emph{semantical} model, provided the syntactic constructors
  correspond to semantic ones in the model, whereas the other rules
  should be proved before being introduced.}.

For example, to see the program constant \texttt{NatPlus} and its rules
type:
\begin{verbatim}
(display-pconst "NatPlus")

> NatPlus
  comprules
        nat+0        nat
        nat1+Succ nat2        Succ(nat1+nat2)
  rewrules
        0+nat        nat
        Succ nat1+nat2        Succ(nat1+nat2)
        nat1+(nat2+nat3)      nat1+nat2+nat3
\end{verbatim}

Note that here \texttt{nat} is a default variable of type nat.  We
recommend to have a look at the file \texttt{nat.scm} to familiarise
oneself with the way program constants are defined.

To see the effect of term rewriting rules for \texttt{+} we type
\begin{verbatim}
(pp (nt (pt "3+4")))
(pp (nt (pt "Succ n+Succ m+0")))
\end{verbatim}
which yields as results the number \texttt{7} and
\texttt{Succ(Succ(n+m))}.  Here \texttt{pp} stands for \texttt{pretty
  print} and \texttt{nt} stands for \texttt{normalize term}; this
essentially consists in repeatedly applying\footnote{Term rewriting in
  \mi\ makes use of normalisation-by-evaluation (see
  \cite{minlogman}).} the term rewriting rules until no new term is
obtained.


\subsubsection{Adding new program constants and computation rules}
We now wish to exemplify the introduction of new program constants on
the natural numbers.

Recall that if the file \texttt{nat.scm} is not already
loaded\footnote{Clearly, it is good practice to run a new \mi\ session
  when loading new files which could turn out to be incompatible with
  previously loaded files or previously introduced definitions.}, we
can type:
\begin{verbatim}
(set! COMMENT-FLAG #f)
(libload "nat.scm")
(set! COMMENT-FLAG #t)
\end{verbatim}

We now introduce a new program constant which represents the function
which doubles a natural number.  The command used to introduce a new
program constant is \texttt{add-program-constant}. It requires the
name of the constant and its type; further arguments may be the degree
of totality, the token type (e.g.\ \texttt{const}) and the arity (see
\cite{minlogman}).  In particular, note that in \mi\ we can treat not
only total objects but also partial ones\footnote{For the notion of
  totality see \cite[Chapter 8.3]{Stoltenberg94}; see also
  \cite{SchwichtenbergWainer12}.}.  Therefore, when we introduce a new
program constant, we may also specify its totality degree. A totality
degree of one (\texttt{t-deg-one}) indicates that the program constant
is total, while zero (which is the default) denotes non--totality.  As
to the type, in the present case, the new constant \texttt{Double} is
of arrow type, as it takes natural numbers as input and produces
natural numbers as output.

\begin{verbatim}
(add-program-constant "Double" (py "nat=>nat"))
\end{verbatim}

In case we wish to remove this program constant, we simply write:
\begin{verbatim}
(remove-program-constant "Double")
\end{verbatim}

The behaviour of a new program constant can be specified by
introducing one or more computation rules for it.  This is accomplished
by use of the command \texttt{add-computation-rule}, having two
arguments: a left hand side and a right hand side.  The right hand
side specifies the result of the computation rule for the argument
indicated in the left hand side.

The following example should clarify how to use these commands.

The function \inquotes{Double} can be defined by specifying
primitively recursively how it acts on zero and on the successor of
each natural number.

\begin{verbatim}
(add-computation-rule (pt "Double 0") (pt "0"))
(add-computation-rule (pt "Double(Succ n)")
                      (pt "Succ(Succ(Double n))"))
\end{verbatim}

Alternatively, one could also write:
\begin{verbatim}
(add-computation-rules
 "Double 0" "0"
 "Double(Succ n)" "Succ(Succ(Double n))")
\end{verbatim}

To see the effect of the newly introduced computation rules:
\begin{verbatim}
(pp (nt (pt "Double 3")))
(pp (nt (pt "Double(n+2)")))
\end{verbatim}


\subsubsection{Proof by induction}
Here we wish to exemplify a proof by induction on the natural numbers.
The goal is very simple: we wish to show that $\mathtt{Double}\ n=n+n$.
The first step of the proof consists in using the command
\texttt{ind}.  This command requires a universally quantified goal and
proves it by induction, according to the definition of the specific
algebra type.
\begin{verbatim}
(set-goal "all n Double n=n+n")
(ind)
\end{verbatim}

The effect of applying \texttt{ind} is to refine the goal to a proof
of the base and the step cases of the induction.  In the present case,
where the constructors are Zero and Successor, we have to prove two
cases: one for Zero and one for Successor.  \mi's reply will be
something like this:
\begin{verbatim}
ok, ?_1 can be obtained from

  n2218
-------------------------------------------------------
?_3:all n(Double n=n+n -> Double(Succ n)=Succ n+Succ n)

  n2218
-------------------------------------------------------
?_2:Double 0=0+0
\end{verbatim}

We then replace the goal with its normal form by letting:
\begin{verbatim}
(normalize-goal)

  n2218
-------------------------------------------------------
?_4:T
\end{verbatim}

The latter command can be abbreviated with \texttt{ng} and it
normalizes the goal by using the computation rules for
\inquotes{\texttt{+}} introduced in the file \texttt{nat.scm}.  More
specifically, as both Double 0 and 0+0 reduce to 0, the normalization
will first of all produce 0=0.  This in turn reduces to truth, here
indicated by \texttt{T}.  It comes equipped with an axiom
\texttt{Truth}, by means of which we prove the base case.
%% \texttt{T} is a constructor of the
%% predefined algebra of booleans, and it comes equipped with the
%% \texttt{Truth-axiom}, by means of which we prove the base case.
\begin{verbatim}
(use "Truth")

ok, ?_4 is proved.  The active goal now is
  n2218
-------------------------------------------------------
?_3:all n(Double n=n+n -> Double(Succ n)=Succ n+Succ n)

\end{verbatim}

As to the step, we make use of the induction hypothesis, \texttt{IH},
and write:
\begin{verbatim}
(assume "n" "IH")
(ng)
(use "IH")

> ok, we now have the new goal 
  n2218  n  IH:Double n=n+n
-------------------------------------------------------
?_5:Double(Succ n)=Succ n+Succ n
> ok, the normalized goal is
  n2218  n  IH:Double n=n+n
-------------------------------------------------------
?_6:Double n=n+n
> ok, ?_6 is proved.  Proof finished.
\end{verbatim}

Also in this case, when we write \texttt{ng} the term rewriting rules
for \texttt{Double} and \inquotes{\texttt{+}} are applied.

Finally, we wish to recall that one could also define the Double
function without making use of a primitive recursive definition.
\begin{verbatim}
(add-program-constant "DoubleN" (py "nat=>nat"))
(add-computation-rule (pt "DoubleN n") (pt "n+n"))
\end{verbatim}


\subsubsection{Exercises}
Prove that the two definitions of the doubling function are equivalent:
\begin{verbatim}
(set-goal "all n Double n=DoubleN n")
\end{verbatim}
Prove also the following:
\begin{verbatim}
(set-goal "all n,m n+m=m+n")
\end{verbatim}


\subsubsection{Rewrite rules}
Once we have proved the above statement for which the two definitions
of Double are equivalent, we may add a rewrite rule which replaces
each occurrence of \texttt{Double} by \texttt{DoubleN}.
\begin{verbatim}
(add-rewrite-rule (pt "Double n") (pt "DoubleN n"))
\end{verbatim}


\subsubsection{Another example}
\label{EvenOdd}
We now present another example of induction on the natural numbers,
which introduces some additional features of \mi.

Suppose we want to prove that for all natural numbers $n$, $Double\,
n$ is even.  We define two new program constants \texttt{Odd} and
\texttt{Even} which take a natural number as argument and give a
boolean (true or false) as output.  As usual, the behaviour of these
program constants can be specified by means of appropriate computation
rules. In this case the computation rules will simultaneously
characterize \texttt{Odd} and \texttt{Even}.
\begin{verbatim}
(add-program-constant "Odd" (py "nat=>boole"))
(add-program-constant "Even" (py "nat=>boole"))

(add-computation-rules
 "Odd 0" "False"
 "Even 0" "True"
 "Odd(Succ n)" "Even n"
 "Even(Succ n)" "Odd n")
\end{verbatim}

The steps of the proof are self--explanatory:
\begin{verbatim}
(set-goal "all n Even(Double n)")
(ind)
(prop)
(search)
\end{verbatim}


\subsection{Case distinction on the booleans}
We wish to give an example of distinction by cases, and for simplicity
we shall consider a trivial example on the booleans.

We wish to prove that for any boolean $p$, if $p$ is not false then it
is true.  We add a variable \texttt{p} of type boolean and set the
goal:
\begin{verbatim}
(av "p" (py "boole"))
(set-goal "all p((p=False -> F) -> p=True)")
\end{verbatim}

The proof then proceeds by cases: either $p$ is false or it is true.
The following steps should be clear.
\begin{verbatim}
(cases)
(prop)
(prop)
\end{verbatim}


\subsection{Induction on lists}
The following example is an exercise on lists over an arbitrary type
$\alpha$.  This example illustrates again the use of induction;
however, since we now deal with \emph{parametrized algebras} (see
\cite{minlogman,SchwichtenbergWainer12}) the task turns out to be a
bit harder than when working with the algebra of natural
numbers%% \footnote{We recall that an algebra is finitary if every
  %% constructor takes only finitely many arguments. Therefore for any
  %% two elements of a finitary algebra we can decide whether they are
  %% equal.  \mi\ also allows for infinitary free algebras, whose
  %% constructors may take infinitely many arguments. Notice that then
  %% equality is not decidable any more, and hence needs to be
  %% axiomatized.  An example of an infinitary algebra is given by the
  %% countable ordinals.  They can be seen as generated from a nullary
  %% constructor 0, a unary constructor for the successor and a
  %% constructor building the supremum out of a countably infinite list
  %% of ordinals (given by a function from the natural numbers to
  %% ordinals).  }
.

To start with we load the file \texttt{list.scm}, which contains basic
definitions and operations on lists over an arbitrary type
$\alpha$\footnote{ Note that \texttt{list.scm} does require to first
  upload \texttt{nat.scm}.  We recommend to go through the list file
  before working out this example.}.  Then we introduce a function,
$\listrev$, on lists which has the effect of reverting a list. Finally
we prove:
\begin{equation*}
  \forall_{v,w} (\eqd{\listrev\, (v \listappend w)}
  {(\listrev\, w) \listappend (\listrev\, v)}),
\end{equation*}
where $v$ and $w$ are lists over an arbitrary type $\alpha$ and $\ast$
denotes the append function on lists as defined in \texttt{list.scm}.
Further, $\equiv$ represent Leibniz' equality\footnote{Internally
  Leibniz equality is printed \texttt{eqd}, where the \texttt{d}
  stands for \inquotes{defined}, since Leibniz equality is inductively
  defined by the clause \texttt{InitEqD}: $\allnc_x \eqd{x}{x}$; see
\ref{EvenI} for inductively defined predicates.}: two
elements are equal if they have the same properties, i.e., they are
indistinguishable.

We begin as follows:
\begin{verbatim}
;; (libload "nat.scm")
(set! COMMENT-FLAG #f)
(libload "list.scm")
(set! COMMENT-FLAG #t)

(add-var-name "x" "a" "b" "c" "d" (py "alpha"))
(add-var-name "xs" "v" "w" "u" (py "list alpha"))
\end{verbatim}

We now need to define $\listrev$.  This is defined inductively, by
first giving its value for the empty list and then saying how it
applies to a non-empty list.  The two defining conditions for
$\listrev$ are the following:
\begin{align*}
\listrev\,(\nil\, \alpha) &= (\nil\, \alpha),
\\
\listrev\,(a::w) &= (\listrev\, w) \listappend (a{:})
\end{align*}
where, according to the notation in \texttt{list.scm}, $(\nil\, \alpha)$
denotes the empty list over the type $\alpha$, $a::w$ denotes the list
obtained by adding the object $a$ of type $\alpha$ to the list $w$
(over $\alpha$), while $a{:}$ is the one element list obtained from
$a$.  We thus write:
\begin{verbatim}
(add-program-constant "ListRv"
            (py "list alpha => list alpha") t-deg-one)
(add-prefix-display-string "ListRv" "Rv")

(add-computation-rules
 "Rv(Nil alpha)" "(Nil alpha)"
 "Rv(x::xs)" "Rv xs++x:")
\end{verbatim}

Note that for simplicity we have stated that \texttt{ListRv} is a
total function.  \mi's output will include a warning, to remind us
that we should have separately proved before that \texttt{ListRv} is
in fact total.  Also, \texttt{add-prefix-display-string} allows us to
define a token for the program constant\footnote{In the file
  \texttt{list.scm} there already exists a program constant
  \texttt{ListRev} with display string \texttt{Rev} defined exactly as
  our \texttt{ListRv}.  Here we have just duplicated the definition
  for pedagogical reasons, since it is a nice and easy example.
  However, we had to choose a different name to avoid a clash with the
  already loaded \texttt{list.scm}.}.

The following proof makes use of a program constant,
\texttt{ListAppd}, already available within the file
\texttt{list.scm}.  This has the following computation rules:
\begin{verbatim}
         (Nil alpha)++xs2      xs2
         (x1::xs1)++xs2        x1::xs1++xs2
\end{verbatim}
And rewrite rules:
\begin{verbatim}
         xs++(Nil alpha)       xs
         xs1++x2: ++xs2        xs1++(x2::xs2)
\end{verbatim}

To check \texttt{ListAppd} we type:
\begin{verbatim}
(display-pconst "ListAppd")
\end{verbatim}

Now we can set the goal and start the proof by calling \texttt{ind}:
\begin{verbatim}
(set-goal "all v,w Rv(v++w)eqd Rv w++Rv v")
(ind)
\end{verbatim}

This has the effect of producing two subgoals, corresponding to the
base case and the step case, respectively.  We tackle the base case as
follows:
\begin{verbatim}
(ng)
(assume "w")
(use "InitEqD")
\end{verbatim}
Here we have used \texttt{InitEqD}, which is the axiom: \texttt{xs eqd
  xs}.

Subsequently we move to the step case:
\begin{verbatim}
(assume "a" "v" "IHw" "w")
(ng)
(simp "IHw")
\end{verbatim}

And finally we use a theorem proved in the file \texttt{list.scm} and
there called \texttt{ListAppdAssoc:}
\begin{verbatim}
all xs1,xs2,xs3 xs1++(xs2++xs3)eqd xs1++xs2++xs3
\end{verbatim}
Then we carry on by
\begin{verbatim}
(simp "ListAppdAssoc")
(use "InitEqD")
\end{verbatim}

\subsubsection{Exercise}
In \texttt{list.scm} \texttt{ListMap} is introduced by
\begin{verbatim}
(add-program-constant
 "ListMap" (py "(alpha1=>alpha2)=>list alpha1=>list alpha2"))

(add-infix-display-string "ListMap" "map" 'pair-op)

(add-var-name "phi" (py "alpha1=>alpha2"))

(add-computation-rules
 "phi map(Nil alpha1)" "(Nil alpha2)"
 "phi map y::ys" "phi y::phi map ys")
\end{verbatim}
Prove that \verb|map| commutes with \texttt{Rv}:
\begin{verbatim}
(av "f" (py "alpha=>alpha"))
(set-goal "all f,xs (f map Rv xs)eqd Rv(f map xs)")
\end{verbatim}
In the proof it is helful to use the theorem \texttt{MapAppd}, which
can be found in \texttt{list.scm}.

\subsection{Defining algebras: binary trees}
We now wish to show how to introduce new algebras; we shall also give
one more example on how to use them.  The example we shall consider is
that of binary trees.  First of all we introduce a new algebra, called
\inquotes{bintree} which has constructors \inquotes{Null} and
\inquotes{Con}.  We also add two variables of type \inquotes{bintree}:
\inquotes{ltree} and \inquotes{rtree} (for left and right tree).
\begin{verbatim}
(add-algs "bintree"
          '("bintree" "Null")
          '("bintree=>nat=>bintree=>bintree" "Con"))
(av "ltree" "rtree" (py "bintree"))
\end{verbatim}

We then add a new program constant (by using a shortcut, \texttt{apc},
for the command \texttt{add-program-constant}) and its respective
computation rules.  \texttt{Flatten} takes a tree and produces a list
consisting of the labels in the tree, starting with the root label.
\begin{verbatim}
(apc "Flatten" (py "bintree=>list nat"))
(add-computation-rules
 "Flatten(Null)" "(Nil nat)"
 "Flatten(Con ltree n rtree)"
 "n: ++Flatten ltree++Flatten rtree")
\end{verbatim}

To see how this works, one can for example type:
\begin{verbatim}
(pp (nt (pt "Flatten(Con(Con Null 4 Null)
                        1
                        (Con Null 5(Con Null 7 Null)))")))
\end{verbatim}
to obtain \texttt{1::4::5::7:} as the list of labels.


\subsection{Inductively defined predicates}
\label{EvenI}
In \mi\ we can also introduce inductively generated predicates with
the command \texttt{add-ids}.  An example defining the even numbers
inductively is as follows:
\begin{verbatim}
(add-ids
  (list (list "EvenI" (make-arity (py "nat")) "algEvenI"))
  '("EvenI 0" "InitEvenI")
  '("allnc n(EvenI n -> EvenI(n+2))" "GenEvenI"))
\end{verbatim}
The two closure axioms of this inductive definition are
\texttt{InitEvenI:} \texttt{EvenI 0} and \texttt{GenEvenI: allnc n(
  EvenI n -> EvenI(n + 2))}.  In the latter, we have made use of the
\inquotes{non-computational} quantifier \texttt{allnc}%
\index{allnc@\texttt{allnc}}.  We briefly recall that a
non-computational quantifier may be used in cases where the variable
it quantifies on will not be used (freely) in any term in the
proof.\footnote{Consequently, extracted programs (see the next
  chapter) will not depend on these variables.}  Seen purely logically
there is no difference between the \texttt{all} and \texttt{allnc}
quantifier.  With \texttt{algEvenI}, we provide a name for an algebra,
corresponding to this inductive definition.  It can be omitted, in
which case we have an \inquotes{inductive definition without
  computational content}.

Similarly to the case of simultaneous free algebras, we could use
\texttt{add-ids} to introduce simultaneously more than one predicate.
The case above is a particular example, in which the first occurrence
of \texttt{list} is followed by only one item.

Let's now see a proof which uses the closure axioms for
\texttt{EvenI}.  This makes essential use of the command
\texttt{intro}\index{intro@\texttt{intro}}.  The command
\texttt{(intro i .\ terms)} expects as goal a formula which can be
proved using the i-th closure axiom.  This axiom is then applied, via
\texttt{use}, hence \texttt{terms} may have to be provided.  For
example, below by calling \texttt{intro 0} we apply the first closure
axiom for EvenI, and by calling \texttt{intro 1} we apply the second
closure axiom.
\begin{verbatim}
(set-goal "all n EvenI(n+n)")
(ind)
(ng)
(intro 0)
(assume "n" "IH")
(ng)
(intro 1)
(use "IH")
\end{verbatim}

To conclude this section, we present a proof using induction on the
predicate \texttt{EvenI}.
\begin{verbatim}
(set-goal "allnc n(EvenI n -> ex m m+m=n)")
(assume "n" "En")
(elim "En")
\end{verbatim}
Here \texttt{elim}\index{elim@\texttt{elim}} applies the induction
axiom for the inductive definition.  For example, if the goal is
\texttt{all n(EvenI n -> P n)}, we obtain the new goals: \texttt{P 0}
and \texttt{all n((EvenI n \& P n) -> P(n+2))}.  The commands for the
rest of the proof have been explained before.
\begin{verbatim}
(ex-intro (pt "0"))
(use "Truth")

(assume "n1" "En1" "IH")
(by-assume "IH" "m0" "m0Prop")
(ex-intro (pt "m0+1"))
(simp "<-" "m0Prop")
(use "Truth")
\end{verbatim}

%% The following exercise seems to be too difficult.
%% As an exercise, we recommend to prove the above statement using a
%% simultaneous inductive definition of Even/Odd.  The simultaneous
%% definition and a hint of how to apply the \verb|elim| command can be
%% found in the reference manual.

\subsection{Totality of program constants}
\label{SS:Totality}
An important example for an inductively defined predicate is the
totality predicate for an algebra, for instance \texttt{TotalNat}%
\index{TotalNat@\texttt{TotalNat}} for the algebra \texttt{nat}.  It
can be created by calling \texttt{(add-totality \textsl{alg-name})}%
\index{add-totality@\texttt{add-totality}}.  When the file
\texttt{nat.scm} is loaded this already has been done and
\texttt{TotalNat} is inductively defined by the two clauses
\texttt{TotalNatZero} and \texttt{TotalNatSucc}:
\begin{align*}
  &\mathtt{TotalNatZero} \colon	\mathtt{TotalNat}\; 0
  \\
  &\mathtt{TotalNatSucc} \colon
  \allnc_{\hat{n}}(\texttt{TotalNat}\; \hat{n} \to
  \texttt{TotalNat}(\texttt{Succ}\; \hat{n}))
\end{align*}
This can be checked by executing
\begin{verbatim}
(display-idpc "TotalNat")
\end{verbatim}
Note that here we have used $\hat{n}$ rather than $n$ as a variable
name.  At this point it is appropriate to remember that in our
intended model (the Scott-Ershov partial continuous functionals)
\emph{partial} objects are first class citizens, and hence quantifiers
by default range over them.  When we want to talk about total objects
only, we need to relativize quantifiers to a totality predicate.  To
make the notation less cumbersome we introduce the convention that if
a variable name is followed by a \verb#^#, a general (or
partial\index{partial}\index{variable!partial})) variable is meant.
Variable names without a \verb#^# are implicitly restricted to range
over total objects only.  In fact, $\forall_x P x$ is just a
convenient abbreviation for $\allnc_{\hat{x}}(\GTotal \hat{x} \to P
\hat{x})$.  Here $\GTotal$ is the totality predicate for the current
type.

When adding a program constant and its computation rules the default
is that this constant denotes a partial functional.  However,
often the computation rules are such that it actually is total (i.e.,
defined for all total arguments).  It is good practice to prove
totality immediately after defining a program constant.

For example, the program constant \texttt{Double} clearly is total.
To let \mi\ know this fact we have to prove a lemma.
\begin{verbatim}
;; DoubleTotal
(set-goal (rename-variables
            (term-to-totality-formula (pt "Double"))))
\end{verbatim}
The goal then is, as expected,
\begin{verbatim}
?_1: allnc n^(TotalNat n^ -> TotalNat(Double n^))
\end{verbatim}
Now we carry on by assuming the variable and the hypothesis
\begin{verbatim}
(assume "n^" "Tn")
\end{verbatim}
At this point we use the elimination axiom for \texttt{TotalNat}
\begin{verbatim}
(elim "Tn")
(use "TotalNatZero")
(assume "n^1" "Tn1" "IH")
(ng #t)
(use "TotalNatSucc")
(use "TotalNatSucc")
(use "IH")
;; Proof finished.
(save "DoubleTotal")
\end{verbatim}
It is important to give this lemma the name \texttt{DoubleTotal},
i.e., the name of the program constant followed by \texttt{Total}.
After the lemma with this name is saved \mi\ will know that
\texttt{Double} is total.

\section{Program extraction from proofs}
\label{S:RevI}
In this section we give some basic examples of program extraction from
proofs.  We should perhaps mention at this point that program
extraction was one of the main original motivations in the development
of \mi.  In addition, \mi\ features some interesting aspects, as for
example it implements a refined version of the so--called
$A$--translation, thus allowing for program extraction from
\textbf{classical} proofs.  An exposition of program extraction from
proofs and (modified) $A$--translation is well beyond the purpose of
this tutorial.  See for example \cite{BergerBuchholzSchwichtenberg02,%
  BenlBergerSchwichtenbergSeisenbergerZuber98,%
  SchwichtenbergWainer12}.

\subsection{List reversal}
\label{SS:ListReversal}
For our first example of program extraction from proofs, we introduce
an inductively defined predicate \texttt{RevI} (without computational
content) as follows:
\begin{verbatim}
;; (set! COMMENT-FLAG #f)
;; (libload "nat.scm")
;; (libload "list.scm")
;; (set! COMMENT-FLAG #t)

;; (add-var-name "a" "b" "c" "d" "x" (py "alpha"))
;; (add-var-name "v" "w" "u" "xs" (py "list alpha"))

(add-ids
 (list (list "RevI"(make-arity (py "list alpha")
                               (py "list alpha"))))
 '("RevI(Nil alpha)(Nil alpha)" "InitRevI")
 '("all a,v,w(RevI v w  -> RevI(v++a:)(a::w))" "GenRevI"))
\end{verbatim}
%% (add-ids
%%  (list (list "RevI"(make-arity (py "list alpha")
%%                                (py "list alpha"))))
%%  '("RevI(Nil alpha)(Nil alpha)" "InitRev")
%%  '("all a,v,w(RevI v w  -> RevI(a::v)(w++a:))" "GenRevI"))
%% \end{verbatim}
We first prove that \texttt{RevI} satisfies a variant
clause\footnote{We did not take this variant as the defining clause
  since our definition will be useful in section~\ref{SS:ExtrClass}.}:
\begin{verbatim}
(set-goal "all a,v,w(RevI v w  -> RevI(a::v)(w++a:))")
(assume "a" "v" "w" "Rvw")
(elim "Rvw")
(ng)
;; RevI(a:)(a:)
(use-with "GenRevI" (pt "a") (pt "(Nil alpha)") (pt "(Nil alpha)")
          "InitRevI")
(assume "a1" "v1" "w1" "Rv1w1" "Hyp")
(assert "(a::v1++a1:)eqd(a::v1)++a1:")
 (ng #t)
 (use "InitEqD")
(assume "Assertion1")
(simp "Assertion1")
(assert "(a1::w1)++a:eqd(a1::w1++a:)")
 (ng #t)
 (use "InitEqD")
(assume "Assertion2")
(simp "Assertion2")
(use "GenRevI")
(use "Hyp")
;; Proof finished.
(save "RevIConsAppd")
\end{verbatim}
Using \texttt{RevIConsAppd} we can prove symmetry of \texttt{RevI}
\begin{verbatim}
(set-goal "all v,w(RevI v w -> RevI w v)")
(assume "v" "w" "Rvw")
(elim "Rvw")
(use "InitRevI")
(assume "a" "v1" "w1" "Rv1w1" "Rw1v1")
(use "RevIConsAppd")
(use "Rw1v1")
;; Proof finished.
(save "RevISym")
\end{verbatim}
Then we set the goal:
\begin{verbatim}
(set-goal "all v ex w RevI v w")
\end{verbatim}
The proof proceeds by structural induction on lists.  We first call
\texttt{ind}.  Subsequently, we tackle the base case by first of all
providing a witness, \texttt{Nil} of type $\alpha$, and then by using
the first closure axiom for \texttt{RevI}.
\begin{verbatim}
(ind)
(ex-intro (pt "(Nil alpha)"))
(intro 0)
\end{verbatim}

The step is proved by first of all calling some standard commands
(\texttt{assume, by-assume}), then by providing a witness, and
finally the second closure axiom is called as usual by an \texttt{intro}
command.
\begin{verbatim}
(assume "a" "v" "IH")
(by-assume "IH" "w" "wProp")
(ex-intro (pt "w++a:"))
(use "RevISym")
(intro 1)
(use "RevISym")
(use "wProp")
\end{verbatim}

We can finally name the proof we have just completed by writing:
\begin{verbatim}
(define constr-proof (current-proof))
\end{verbatim}

Note that \texttt{(current-proof)} stores the latest proof.

We now extract a program from the proof and normalize it as follows:
\begin{verbatim}
(define eterm (proof-to-extracted-term constr-proof))
(define neterm (rename-variables (nt eterm)))
(pp neterm)
\end{verbatim}

We have used \texttt{rename-variables} to obtain a more readable term.
This \inquotes{normalized extracted term} \texttt{neterm} is the
program we are looking for.  To display it we write:
\begin{verbatim}
(pp neterm)
\end{verbatim}

The output will be:
\begin{verbatim}
[xs](Rec list alpha=>list alpha)xs
    (Nil alpha)([x,xs0,xs1]xs1++x:)
\end{verbatim}

Here \texttt{[xs]} denotes abstraction on the variable \texttt{xs},
usually also written by use of the $\lambda$ notation.  We observe
that the extracted term uses the recursion operator \texttt{Rec}.  In
more familiar terms, it amounts to a program, which we may call
\texttt{Reverse}, defined as follows:
\begin{verbatim}
Reverse Nil=Nil
Reverse (x :: xs)=(Reverse xs) ++ x:
\end{verbatim}

%% We now extract a program from the proof and normalize it as follows:
%% \begin{verbatim}
%% (define program
%%    (nt (proof-to-extracted-term constr-proof)))
%% \end{verbatim}

%% To display the program we write:
%% \begin{verbatim}
%% (pp program)
%% \end{verbatim}

%% The output will be:
%% \begin{verbatim}
%% [xs0](Rec list alpha=>list alpha)xs0
%%      (Nil alpha)([x1,xs2,xs3]xs3++x1:)
%% \end{verbatim}

%% Here \texttt{[xs0]} denotes abstraction on the variable \texttt{xs0},
%% usually also written by use of the $\lambda$ notation.  We observe
%% that the extracted term uses the recursion operator \texttt{Rec}.

Note that we could have also displayed the program by using the
command \texttt{term-to-scheme-expr}, which produces the
$\lambda$-term corresponding to the program.

To test the program we can \inquotes{run} it on input \texttt{[a,b,c,d]}:
\begin{verbatim}
(pp (nt (make-term-in-app-form neterm (pt "a::b::c::d:"))))
\end{verbatim}
and obtain the result: \texttt{d::c::b::a:}

\subsection{Program extraction from proofs using inductive
  definitions with computational content} In this section we wish to
exemplify how to extract a program from a proof by induction of a
statement which uses an inductive definition\footnote{These inductive
  definitions are also called \inquotes{inductive definitions with
    computational content}. Note that there are also inductive
  definitions without computational content (for example \texttt{RevI}
  above).  For more information on computational content, see for
  example \cite{SchwichtenbergWainer12} and
  \cite{BergerSeisenberger10}.}.  In fact, an inductive definition
with computational content on the proof side corresponds to a free
algebra on the program side.  Thus we need to provide a name for such
an algebra, if a new one has to be generated.  However, if there is an
already existing algebra with fitting constructors, the name of this
algebra can be provided as well.  I.e., in our example on the even
numbers, which we recall below, we could also take \texttt{nat}.  The
types of the algebra's constructors corresponds to the clauses of the
inductive definition, named \texttt{InitEven} and \texttt{GenEven}.
\begin{verbatim}
(add-ids
 (list (list "EvenI" (make-arity (py "nat")) "algEvenI"))
 '("EvenI 0" "InitEvenI")
 '("allnc n(EvenI n -> EvenI(n+2))" "GenEvenI"))
\end{verbatim}
We recall our proof and extract a program:
\begin{verbatim}
(set-goal "allnc n(EvenI n -> ex m m+m=n)")
(assume "n" "En")
(elim "En")
(ex-intro (pt "0"))
(use "Truth")
(assume "n1" "En1" "IH")
(by-assume "IH" "m0" "m0Prop")
(ex-intro (pt "m0+1"))
(simp "<-" "m0Prop")
(use "Truth")

(define eterm (proof-to-extracted-term (current-proof)))
(define neterm (rename-variables (nt eterm)))
\end{verbatim}
We can see the program by writing:
\begin{verbatim}
(pp neterm)
\end{verbatim}
Here, the extracted program essentially corresponds to the identity
function.  For comparison we recommend to extract a program using the
earlier defined program constants \verb|Even/Odd|.
\begin{verbatim}
(set-goal "all n ex m((Even n -> 2*m=n) &
                      (Odd n -> 2*m+1=n))")
\end{verbatim}
One first has to prove totality of \verb|Even| and \verb|Odd|, then
prove their disjointness and finally prove the goal above.  A solution
to this exercise can be found in Appendix~\ref{SS:EvenOdd}.

As mentioned, we can define the even predicate by an inductive
definition without computational content.
\begin{verbatim}
(add-ids (list (list "EvenNC" (make-arity (py "nat"))))
	 '("EvenNC 0" "InitEvenNC")
	 '("allnc n(EvenNC n -> EvenNC(n+2))" "GenEvenNC"))
\end{verbatim}
Generally, for non-computational (n.c.)\ inductively defined
predicates no algebra name is provided.  Important special cases are
the \inquotes{uniform one clause defined}%
\index{uniform one clause defined} inductive predicates\footnote{ An
  inductive predicates is called uniform one clause defined if it has
  just one clause with $\allnc$ and $\tonc$ (an n.c.\ variant of
  $\to$) only; see the reference manual for details.}
which are non-computational.  Examples are Leibniz equality%
\index{Leibniz equality} \texttt{EqD}\index{EqD}, and uniform variants
\texttt{ExU}\index{ExU} and \texttt{AndU}\index{AndU} of the
existential quantifier and conjunction.

One can prove a non-computational version of the lemma above, where
we only need to replace the \texttt{ex-intro} command by the general
\texttt{intro}, since \texttt{ExU} is inductively defined.
\begin{verbatim}
(set-goal "allnc n(EvenNC n -> exu m m+m=n)")
(assume "n" "En")
(elim "En")
(intro 0 (pt "0"))
(use "Truth")
(assume "n1" "En1" "IH")
(by-assume "IH" "m0" "m0Prop")
(intro 0 (pt "m0+1"))
(simp "<-" "m0Prop")
(use "Truth")
\end{verbatim}

%% \subsection{Another example}
%% In this section we wish to exemplify how to extract a program from a
%% proof by induction of a statement which uses an inductive definition.
%% The inductive definition we wish to implement has computational
%% content\footnote{These inductive definitions are also called
%%   \inquotes{inductive definitions with computational content}. Note
%%   that there are also inductive definitions without computational
%%   content (for example \texttt{RevI} above).
%% %In the latter case, the inductive definition is just used as a
%% %predicate which (roughly speaking) does not occur on the program side.
%% For more on computational content, see for example
%% \cite{SchwichtenbergWainer12} and \cite{BergerSeisenberger10}.}; thus
%% we define an algebra corresponding to it.  In fact, an inductive
%% definition with computational content on the proof side corresponds to
%% a free algebra on the program side. Thus we need to provide a name for
%% such an algebra, if a new one has to be generated.  However, if there
%% is an already existing algebra with fitting constructors, the name of
%% this algebra can be provided as well.  I.e., in the example below we
%% could alse take \texttt{nat}.  The types of the algebra's constructors
%% corresponds to the clauses of the inductive definition, named
%% \texttt{InitEven} and \texttt{GenEven}.
%% %% as well as names for the constructors.  We call
%% %% \texttt{algEven} the algebra and \texttt{InitEven} and
%% %% \texttt{GenEven}, respectively, the two constructors.  The latter
%% %% correspond to the closure axioms of the inductive definition.
%% \begin{verbatim}
%% (add-ids
%%  (list (list "EvenII" (make-arity (py "nat")) "algEvenII"))
%%  '("EvenII 0" "InitEvenII")
%%  '("allnc n(EvenII n -> EvenII(n+2))" "GenEvenII"))
%% \end{verbatim}

%% In the following we shall also make use of the
%% \inquotes{non--computational} quantifier \texttt{allnc}.  Here we
%% briefly recall that a \inquotes{non--computational} quantifier may be
%% used in this case as the variable it quantifies on will not be used
%% free in a term later on in the proof. This is to say, the extracted
%% program doesn't depend on this variable.

%% We set the following goal:
%% \begin{verbatim}
%% (set-goal "allnc n(EvenII n -> ex m m+m=n)")
%% \end{verbatim}

%% The proof proceeds as follows:
%% \begin{verbatim}
%% (assume "n")
%% (elim)
%% (ex-intro (pt "0"))
%% (prop)
%% (assume "n1" 2 3)
%% (by-assume 3 "m0" 4)
%% (ex-intro (pt "m0+1"))
%% (simp "<-" 4)
%% (use "Truth-Axiom")
%% \end{verbatim}

%% We then give a name to the proof just completed:
%% \begin{verbatim}
%% (define eterm (proof-to-extracted-term (current-proof)))
%% (define neterm (rename-variables (nt eterm)))
%% \end{verbatim}

%% We can see the program by writing:
%% \begin{verbatim}
%% (pp neterm)
%% \end{verbatim}

We conclude this section with a more substantial example of program
extraction from proofs involving inductive definitions.  Every
constructive proof of an existential theorem (or \inquotes{problem};
cf.\ \cite{Kolmogorov32}) contains -- by the very meaning of
\inquotes{constructive proof} -- a construction of a solution in terms
of the parameters of the problem.  To get hold of such a solution we
have two methods.

\emph{Write-and-verify}.  Guided by our understanding of how the
constructive proof works we directly write down a program to compute
the solution, and then formally prove (\inquotes{verify}) that this
indeed is the case.

\emph{Prove-and-extract}.  Formalize the constructive proof, and then
extract the computational content of this proof in the form of a
realizing term $t$.  The soundness theorem guarantees (and even
provides a formal proof) that $t$ is a solution to the problem.

In simple cases the two methods are often essentially the same.
However, in more complex situations the prove-and-extract method seems
to be preferable, for the following reasons.
\begin{enumeratei}
\item Dealing with a problem on the proof level makes it possible to
  use more abstract mathematical tools.
\item Generally a better organization of the material becomes
  possible, which is an essential aspect of a good mathematical analysis
  of a problem.
\item Such a structural approach leads to a better understanding of
  what is going on, which will make it easier to adapt the proof to a
  somewhat changed specification.
\end{enumeratei}
Consider the problem of recognizing whether a list of left and right
parentheses is balanced, and if so produce a generating tree
(a.k.a.\ parse tree).  Usually one tackles this problem by the
write-and-verify method: one writes such a parser as a shift-reduce
syntax analyser, and verifies that it is correct and complete.  But we
can view it also as a good test for the prove-and-extract method.
However, since this example is rather complicated we have relegated
its treatment into Appendix~\ref{SS:Parsing}.

\subsection{Program extraction from classical proofs}
\label{SS:ExtrClass}
Finally, we wish to exemplify how to extract programs from classical
proofs.  Once more, an account of the theory underlying this example
exceeds the modest aims of this tutorial, so that we can but refer the
inquisitive reader to the literature already mentioned above.

The goal is to prove a classical variant of the statement in
Section~\ref{SS:ListReversal}.  Quite concisely, we set the goal and
produce a proof:
\begin{verbatim}
(set-goal "all v excl w RevI v w")

(assume "v0" "AllNegHyp")
(cut "all u allnc v(v++u eqd v0 ->
                   all w(RevI v w -> bot))")

(assume "claim")
\end{verbatim}

Now we can make use of the \texttt{claim} to prove the goal:
\begin{verbatim}
(use "claim"
     (pt "v0") (pt "(Nil alpha)") (pt "(Nil alpha)"))
(ng)
(use "InitEqD")
(intro 0)
\end{verbatim}

And prove the \texttt{claim} by induction:
\begin{verbatim}
(ind)
\end{verbatim}

The base case is tackled as follows:
\begin{verbatim}
(assume "v")
(ng)
(assume "v=v0" "w")
(simp "v=v0")
(use "AllNegHyp")
\end{verbatim}

As to the step we write:
\begin{verbatim}
(assume "a" "u" "IH" "v" "EqDHyp" "w" "RHyp")
(use "IH" (pt "v++a:") (pt "a::w"))
(ng)
(use "EqDHyp")
(intro 1)
(use "RHyp")
\end{verbatim}

Finally we name the proof, which we conveniently call \texttt{class-proof}:
\begin{verbatim}
(define class-proof (np (current-proof)))
\end{verbatim}
Finally we add a variable \texttt{g}:
\begin{verbatim}
(av "g" (py "list alpha=>list alpha"))
\end{verbatim}
The reasons for this are purely cosmetic. In fact, \texttt{g} will be
the default name in case the extracted program needs a variable of
type \texttt{list alpha=>list alpha}.  Otherwise the program would use
a default variable.
\begin{verbatim}
(define eterm
  (atr-min-excl-proof-to-structured-extracted-term
   class-proof))
(define neterm (rename-variables (nt eterm)))
\end{verbatim}
%% Note that this contains still a term \texttt{cEqDCompatRev} which
%% denotes a lemma used in the above proof (when we called the command
%% \texttt{simp}).

%% In general, if we extract a program from a proof which makes use of a
%% lemma that we previously proved, then the proof will contain the name
%% of such lemma.  We have two possible ways of eliminating such
%% reference to the lemma.
%% \begin{itemize}
%% \item Replace the name of the lemma by its actual proof.  In the
%%   present context, this could be achieved by executing the command:
%%   \texttt{(expand-theorems (current-proof))}.
%% \item Alternatively, on the program side, we can replace the name of
%%   the lemma by the name of a corresponding subprogram.  This is
%%   achieved by use of the command \texttt{animate}, which has the
%%   effect of adding a computation rule.
%% \end{itemize}

%% In the present case, we replace \texttt{cEqDCompatRev} by its program:
%% \begin{verbatim}
%% (animate "EqDCompatRev")
%% \end{verbatim}

%% This will produce the output:
%% \begin{verbatim}
%% ok, computation rule (cEqDCompatRev alpha7)
%%         -> [alpha7_0]alpha7_0 added
%% \end{verbatim}

We display the program and obtain the output:
\begin{verbatim}
(pp neterm)

[xs]
 (Rec list alpha=>list alpha=>list alpha)xs([xs0]xs0)
 ([x,xs0,g,xs1]g(x::xs1))
 (Nil alpha)
\end{verbatim}

Finally:
\begin{verbatim}
(pp (nt (make-term-in-app-form neterm (pt "a::b::c:"))))
\end{verbatim}
This gives the result:
\begin{verbatim}
c::b::a:
\end{verbatim}

To conclude, we would like to remark that this program differs from
that obtained in section~\ref{SS:ListReversal}.  In fact, the program
above could be written in a more readable form as follows:
\begin{verbatim}
Reverse xs0 = reverse-acc xs0 Nil

reverse-acc Nil xs1 = xs1
reverse-acc (x1::xs2) xs4 = reverse-acc xs2 (x1::xs4)
\end{verbatim}
The reader can see that this program is linear, hence better than the
previous one which is quadratic.

\newpage
\begin{appendix}

\section{Extraction examples}
\label{S:Extraction}

\subsection{Even and Odd}
\label{SS:EvenOdd}
We first need to prove totality of Even and Odd
\begin{verbatim}
;; NatEvenOddTotal
(set-goal "allnc n^(TotalNat n^ -> TotalBoole(Even n^) &
                                   TotalBoole(Odd n^))")
(assume "n^" "Tn")
(elim "Tn")
(split)
(use "TotalBooleTrue")
(use "TotalBooleFalse")
(assume "n^1" "Tn1" "IHn1")
(split)
(ng #t)
(use "IHn1")
(ng #t)
(use "IHn1")
;; Proof finished.
(save "NatEvenOddTotal")

;; EvenTotal
(set-goal
 (rename-variables (term-to-totality-formula (pt "Even"))))
(assume "n^" "Tn")
(use "NatEvenOddTotal")
(use "Tn")
;; Proof finished.
(save "EvenTotal")

;; OddTotal
(set-goal
 (rename-variables (term-to-totality-formula (pt "Odd"))))
(assume "n^" "Tn")
(use "NatEvenOddTotal")
(use "Tn")
;; Proof finished.
(save "OddTotal")
\end{verbatim}
Next we prove that Even and Odd are disjoint.
\begin{verbatim}
;; NatEvenOddDisjunct
(set-goal "all n(Even n -> Odd n -> F)")
(ind)
;; Base
(ng)
(assume "Useless" "Absurd")
(use "Absurd")
;; Step
(assume "n" "IHn" "E(n+1)" "O(n+1)")
(use-with "IHn" "O(n+1)" "E(n+1)")
;; Proof finished.
(save "NatEvenOddDisjunct")
\end{verbatim}
Finally we can prove our goal and extract.
\begin{verbatim}
(set-goal
 "all n ex m((Even n -> 2*m=n) & (Odd n -> 2*m+1=n))")
(ind)
;; Base
(ex-intro "0")
(split)
(assume "Useless")
(use "Truth")
(assume "Absurd")
(use "Absurd")
;; Step
(assume "n" "IHn")
(by-assume "IHn" "m" "mProp")
(ex-intro "[if (Even n) m (Succ m)]")
(split)
;; Case Odd n
(assume "E(n+1)")
(assert "Even n -> F")
 (assume "En")
 (use "NatEvenOddDisjunct" (pt "n"))
 (use "En")
 (use "E(n+1)")
(assume "Even n -> F")
(simp "Even n -> F")
(ng #t)
(use "mProp")
(use "E(n+1)")
;; Case Even n
(assume "O(n+1)")
(simp "O(n+1)")
(ng #t)
(use "mProp")
(use "O(n+1)")
;; Proof finished.

(define eterm (proof-to-extracted-term (current-proof)))
(define neterm (rename-variables (nt eterm)))
(pp neterm)
;; [n](Rec nat=>nat)n 0([n0,n1][if (Even n0) n1 (Succ n1)])
\end{verbatim}

\subsection{Parsing}
\label{SS:Parsing}
Let $E$ range over expressions formed as lists of left and right
parentheses $L,R$.  We are interested in the \cite{Dyck82}\index{Dyck}
language of balanced lists of $L$ and $R$.  It is generated by either
of the grammars
\begin{align*}
  \hbox{grammar U}: &\qquad E \BNFdef \nil \BNFor ELER
  \\
  \hbox{grammar S}: &\qquad E \BNFdef \nil \BNFor LER \BNFor EE
\end{align*}
It is not too difficult to see that both grammars generate the same
expressions.  $S$ appears to be more natural, but its generation trees
are not unique: one can always append the empty list $\nil$.  This can
be repaired easily by only dealing with non-empty lists.  However, a
drawback then is that one often wants to specialize general lemmas
(like the closure poperty of $U$ below) to the empty list.  Therefore
we restrict attention to $U$.

First we formulate the grammar $U$ as an inductively defined predicate
over lists $x,y,z$ of parentheses $L,R$ given by the clauses
\begin{align*}
  & \mathrm{InitU} \colon U (\nil)
  \\
  & \mathrm{GenU} \colon U x \to U y \to U(xLyR)
\end{align*}
The corresponding free algebra on the program side will be that
of binary trees, which we introduce first.
\begin{verbatim}
(add-algs "bin"
	  '("bin" "I")
	  '("bin=>bin=>bin" "C"))
\end{verbatim}
Since we will work with lists of parentheses, we need the library file
\verb|list.scm|.
\begin{verbatim}
(set! COMMENT-FLAG #f)
(libload "nat.scm")
(libload "list.scm")
(set! COMMENT-FLAG #t)

(add-algs "par" '("L" "par") '("R" "par"))
(add-totality "par")
\end{verbatim}
As a preparatory step we first have to prove that ordinary equality $=$
(a boolean valued function) implies Leibniz equality, for the two types
\texttt{par} and \texttt{list\ par}.
\begin{verbatim}
;; ParEqToEqD
(set-goal "all par1,par2(par1=par2 -> par1 eqd par2)")
(cases)
(cases)
(assume "Useless")
(use "InitEqD")
(assume "L=R")
(use "EFEqD")
(use "AtomToEqDTrue")
(use "L=R")
(cases)
(assume "R=L")
(use "EFEqD")
(use "AtomToEqDTrue")
(use "R=L")
(assume "Useless")
(use "InitEqD")
;; Proof finished.
(save "ParEqToEqD")

(add-var-name "x" "y" "z" (py "list par"))

;; ListParEqToEqD
(set-goal "all x1,x2(x1=x2 -> x1 eqd x2)")
(ind)
(cases)
(assume "Useless")
(use "InitEqD")
(assume "par1" "x1" "Absurd")
(use "EFEqD")
(use "AtomToEqDTrue")
(use "Absurd")
(assume "par1" "x1" "IH")
(cases)
(assume "Absurd")
(use "EFEqD")
(use "AtomToEqDTrue")
(use "Absurd")
(assume "par2" "x2" "=Hyp")
(ng "=Hyp")
(assert "x1=x2")
 (use "=Hyp")
(assume "x1=x2")
(assert "par1=par2")
 (use "=Hyp")
(assume "par1=par2")
(drop "=Hyp")
(assert "x1 eqd x2")
 (use "IH")
 (use "x1=x2")
(assume "x1 eqd x2")
(assert "par1 eqd par2")
 (use "ParEqToEqD")
 (use "par1=par2")
(assume "par1 eqd par2")
(elim "x1 eqd x2")
(assume "x^3")
(elim "par1 eqd par2")
(assume "par^3")
(use "InitEqD")
;; Proof finished.
(save "ListParEqToEqD")
\end{verbatim}

Now we inductively define a predicate (grammar) $U$ over lists of
parentheses.
\begin{verbatim}
(add-ids
 (list (list "U" (make-arity (py "list par")) "bin"))
 '("U(Nil par)" "InitU")
 '("allnc x,y(U x -> U y -> U(x++L: ++y++R:))" "GenU"))
\end{verbatim}
We work with two predicates $\mathrm{RP}(n,x)$ meaning $U(x R^n)$ and
$\mathrm{LP}(n,y)$ meaning $U(L^n y)$.  For $\mathrm{RP}$ we have an
inductive definition
\begin{align*}
  & \mathrm{RP}(0,\nil)
  \\
  & U z \to \mathrm{RP}(n,x) \to \mathrm{RP}(n+1,xzL)
\end{align*}
We define $\mathrm{RP}$ with a parameter predicate to be substituted
by $U$.
\begin{verbatim}
(add-pvar-name "P" (make-arity (py "list par")))

(add-ids
 (list (list "RP" (make-arity (py "nat") (py "list par"))
        "list"))
 '("RP 0(Nil par)" "InitRP")
 '("allnc n,x,z(P z -> RP n x -> RP(Succ n)(x++z++L:))"
   "GenRP"))
\end{verbatim}
The algebra associated with this definition of \verb|RP| is lists of
parentheses.

$\mathrm{LP}$ can be defined via a boolean valued function with
defining equations
\begin{align*}
  \mathrm{LP}(0,\nil) &= \true
  \\
  \mathrm{LP}(n+1,\nil) &= \false
  \\
  \mathrm{LP}(n,Lx) &= \mathrm{LP}(n+1,x)
  \\
  \mathrm{LP}(0,Rx) &= \false
  \\
  \mathrm{LP}(n+1,Rx) &= \mathrm{LP}(n,x)
\end{align*}
In Minlog this reads
\begin{verbatim}
(add-program-constant "LP" (py "nat=>list par=>boole"))

(add-computation-rules
 "LP 0(Nil par)"       "True"
 "LP(Succ n)(Nil par)" "False"
 "LP n(L::x)"          "LP(Succ n)x"
 "LP 0(R::x)"          "False"
 "LP(Succ n)(R::x)"    "LP n x")
\end{verbatim}
As mentioned above, it is advisable to prove totality of a program
constant immediately after its definition.
\begin{verbatim}
(set-totality-goal "LP")
(assert
 "allnc x^(TotalList x^ -> allnc n^(TotalNat n^ -> TotalBoole(LP n^ x^)))")
 (assume "x^" "Tx")
 (elim "Tx")
 (assume "n^" "Tn")
 (elim "Tn")
 (use "TotalBooleTrue")
 (assume "n^1" "Useless1" "Useless2")
 (use "TotalBooleFalse")
 (assume "par^" "Tpar")
 (elim "Tpar")
 (assume "x^1" "Tx1" "IHx1" "n^" "Tn")
 (ng #t)
 (use "IHx1")
 (use "TotalNatSucc")
 (use "Tn")
 (assume "x^1" "Tx1" "IHx1" "n^" "Tn")
 (elim "Tn")
 (use "TotalBooleFalse")
 (assume "n^1" "Tn1" "Useless")
 (ng #t)
 (use "IHx1")
 (use "Tn1")
(assume "LPTotalAux" "n^" "Tn" "x^" "Tx")
(use "LPTotalAux")
(use "Tx")
(use "Tn")
;; Proof finished.
(save-totality)
\end{verbatim}
Then clearly the following closure property of $U$ holds
\begin{equation*}
  \mathrm{RP}(n,x) \to U(z) \to \mathrm{LP}(n,y) \to U(xzy).
\end{equation*}
One proves by induction on $y$ that the claim holds for all $n$.
\begin{verbatim}
;; ClosureU
(set-goal
 "all y allnc n,x,z((RP (cterm (x^) U x^))n x ->
  U z -> LP n y -> U(x++z++y))")
(ind)
\end{verbatim}
In the base case $y=\nil$ ine uses induction on $\mathrm{RP}(n,x)$.
\begin{verbatim}
(assume "n" "x" "z" "RP n x")
(elim "RP n x")
;; InitRP
(ng #t)
(auto)
;; GenRP
(ng #t)
(assume "n1" "x1" "z1" "Useless1" "Useless2"
 "Useless3" "Useless4" "Absurd")
(use "Efq")
(use "Absurd")
\end{verbatim}
In the step one distinguishes cases on the first character.  In case
$L::y$ use the induction hypothesis for $n+1$.
\begin{verbatim}
(cases)
(ng #t)
(assume "y" "IHy" "n" "x" "z" "RP n x" "U z" "LP(Succ n)y")
(use-with "IHy" (pt "Succ n") (pt "x++z++L:")
 (pt "(Nil par)") "?" "?" "?")
(use "GenRP")
(use "U z")
(use "RP n x")
(use "InitU")
(use "LP(Succ n)y")
\end{verbatim}
In case $R::y$ again use induction on $\mathrm{RP}(n,x)$.  The first
$\mathrm{RP}$ clause uses Efq, the second one the induction hypothesis
on $y$, \verb|GenU| and equality arguments.
\begin{verbatim}
(assume "y" "IHy" "n" "x" "z" "RP n x")
(elim "RP n x")

;; First RP clause
(ng #t)
(assume "U z" "Absurd")
(use "Efq")
(use "Absurd")

;; Second RP clause.  Uses IHy, GenU and equality arguments.
(assume "n1" "x1" "z1" "U z1" "RP n1 x1" "IH" "U z")
(ng #t)
(simp (pf "x1++z1++(L::z)++(R::y)=x1++z1++(L::z)++R: ++y"))
(simp (pf "x1++z1++(L::z)=x1++(z1++(L::z))"))
(simp (pf "x1++(z1++(L::z))++R: =x1++(z1++(L::z)++R:)"))
(use "IHy")
(use "RP n1 x1")
(use-with "GenU" (pt "z1") (pt "z") "U z1" "U z")
(simp "ListAppdAssoc")
(simp "ListAppdAssoc")
(simp "ListAppdAssoc")
(use "Truth")
(simp "ListAppdAssoc")
(use "Truth")
(ng #t)
(use "Truth")
;; Proof finished
(save "ClosureU")
\end{verbatim}
In particular we have $\mathrm{LP}(0,y) \to U(y)$.

Conversely one can easily prove $U(y) \to \mathrm{LP}(0,y)$ by
induction on $U$.  One needs a property of \verb|LP| first
\begin{verbatim}
;; LPProp
(set-goal "all x,y,n,m(LP n x -> LP m y -> LP(n+m)(x++y))")
(ind)
;; 2,3
(ind)
;; 4,5
(cases)
(cases)
(auto)
;; 5
(ng)
(cases)
(assume "y" "IHy")
(ng)
(assume "n" "m" "Hyp1" "Hyp2")
(use-with "IHy" (pt "n") (pt "Succ m") "Hyp1" "Hyp2")
(assume "y" "IHy" "n")
(cases)
(assume "Hyp1" "Absurd")
(use "Efq")
(use "Absurd")
(ng)
(use "IHy")
;; 3
(cases)
(assume "x" "IHx")
(ng)
(assume "y" "n" "m" "Hyp1" "Hyp2")
(use-with "IHx" (pt "y") (pt "Succ n") (pt "m") "Hyp1" "Hyp2")
(assume "x" "IHx" "y")
(cases)
(assume "m" "Absurd" "Hyp1")
(use "Efq")
(use "Absurd")
(use "IHx")
;; Proof finished.
(save "LPProp")
\end{verbatim}
Using \verb|LPProp| one can prove
\begin{verbatim}
;; Soundness
(set-goal "all y(U y -> LP 0 y)")
(assume "z" "IdHyp")
(elim "IdHyp")
(use "Truth")
(assume "x" "y" "Ux" "LP 0 x" "Uy" "LP 0 y")
(simp "<-" "ListAppdAssoc")
(use-with "LPProp" (pt "x") (pt "L::y++R:") (pt "0") (pt "0")
	  "LP 0 x" "?")
(ng #t)
(use-with "LPProp" (pt "y") (pt "R:") (pt "0") (pt "1")
	  "LP 0 y" "Truth")
;; Proof finished.
(save "Soundness")
\end{verbatim}
From \verb|ClosureU| we obtain 
\begin{verbatim}
;; Completeness
(set-goal "all y(LP 0 y -> U y)")
(assume "y" "LP 0 y")
(use-with "ClosureU" (pt "y") (pt "0")
           (pt "(Nil par)")  (pt "(Nil par)")
         "?" "InitU" "LP 0 y")
(use "InitRP")
;; Proof finished.
(save "Completeness")
\end{verbatim}
Hence the test $\mathrm{LP}(0,y)$ is correct (all $y$ in $U$ satisfies
it) and complete (it implies $y$ in $U$).  Because of
$\mathrm{LP}(0,y) \leftrightarrow U(y)$ we have a decision procedure
for $U$.  With $p$ a boolean variable we can express this by a proof
of
\begin{equation*}
 \forall_y \ex_p((p \to U(y)) \land
                 ((p \to \falsityF) \to U(y) \to \falsityF))
\end{equation*}
\begin{verbatim}
(add-var-name "p" (py "boole"))

;; Parse
(set-goal "all y ex p((p -> U y) & ((p -> F) -> U y -> F))")
(assume "y")
(ex-intro "LP 0 y")
(split)
(use "Completeness")
(assume "LP 0 y -> F" "Uy")
(use "LP 0 y -> F")
(use "Soundness")
(use "Uy")
;; Proof finished.
(save "Parse")
\end{verbatim}
The computational content of this proof is a parser for $U$.  Given
$y$ it returns a boolean saying whether or not $y$ is in $U$, and if
so it also returns a generation tree (a.k.a.\ parse tree) for $U(y)$.

To extract the computational content we need to \inquotes{animate} the
theorems \verb|ClosureU| and \verb|Completeness|, or more precisely
the automatically generated program constants \verb|cClosureU| and
\verb|cCompleteness| abbreviating their computational content.
These constants will be unfolded under normalization once the theorems
are animated.
\begin{verbatim}
(animate "ClosureU")
(animate "Completeness")

(add-var-name "a" (py "bin"))
(add-var-name "as" (py "list bin"))
(add-var-name "f" (py "list bin=>bin=>bin"))

(define eterm
 (proof-to-extracted-term (theorem-name-to-proof "Parse")))
(define neterm-Parse (rename-variables (nt eterm)))
(ppc neterm-Parse)
\end{verbatim}
Here is the term extracted from the proof above.
\begin{verbatim}
[x]LP 0 x@
 (Rec list par=>list bin=>bin=>bin)x
 ([as,a][case as (Nil -> a) (a0::as0 -> I)])
 ([par,x0,f,as,a]
   [case par
     (L -> f(a::as)I)
     (R -> [case as (Nil -> I) (a0::as0 -> f as0(C a0 a))])])
 Nil 
 I
\end{verbatim}
Since this term involves the recursion operator it is not easy to
read.  To grasp its meaning we rewrite it.  It amounts to applying a
function $g$ to $x$, $\nil$ and $I$, where
\begin{alignat*}{2}
  &g(\nil,\as,a) &&=
  \begin{cases}
    a &\hbox{if $\as = \nil$}
    \\
    I &\hbox{else}
  \end{cases}
  \\
  &g(L::x,\as,a) &&=g(x,a::\as,I)
  \\
  &g(R::x,\as,a) &&=
  \begin{cases}
    I &\hbox{if $\as = \nil$}
    \\
    g(x,\as_0,C(a_0,a)) &\hbox{if $\as=a_0::\as_0$}
  \end{cases}
\end{alignat*}
In $g(x,\as,a)$ the first argument $x$ is a list of parentheses $L,R$
to be parsed.  The second argument $\as$ is a stack of parse trees,
and the third $a$ is the working memory of the parser which stores the
parse tree being generated.  Initially $g$ is called with $x$, the
empty stack $\nil$ and the empty parse tree $I$.

Recall the grammar $U$.  We read $x$ from left to right.  When an $L$
occurs, the current parse tree $a$ (corresponding to $E_0$ in $E_0 L
E_1 R$) is pushed onto the stack, and then $g$ starts generating a
parse tree for $E_1$, with the empty parse tree $I$ in its working
memory.  Now suppose $R$ occurs in $x$.  If the stack is $\nil$,
return the empty parse tree $I$.  If not, pop the top element $a_0$
from the stack.  Then $g$ starts generating a parse tree from the rest
of $x$, the tail $\as_0$ of the stack, and as current parse tree
$C(a_0, a)$ in its working memory.

If the input $x$ is empty, in case the stack $\as$ is empty as well
the current parse tree $a$ is returned, and otherwise the empty parse
tree $I$.

To test our extracted \verb|neterm-Parse| we use some simple Scheme
functions.  \verb|(generate-seq n)| generates a list of $2^n$ infinite
sequences starting with all possible variations of $n$ digits and
continuing with $0$.
\begin{verbatim}
(define (generate-seq n)
  (if (= n 0)
      (list (lambda (n) 0))
      (foldr (lambda (f l)
               (cons (lambda (n) (if (= n 0) 0 (f (- n 1))))
                     (cons (lambda (n) (if (= n 0) 1 (f (- n 1))))
                           l)))
             '()
             (generate-seq (- n 1)))))
\end{verbatim}
\verb|(first f n)| returns a list of (f 0),(f 1),...,(f n-1).
\begin{verbatim}
(define (first f n)
  (if (= n 0)
      '()
       (cons (f 0)
             (first (lambda (n) (f (+ n 1))) (- n 1)))))
\end{verbatim}
We also use
\begin{verbatim}
(define (blist-to-lpar-term blist)
  (if (null? blist)
      (pt "(Nil par)")
      (mk-term-in-app-form
       (pt "(Cons par)")
       (if (zero? (car blist)) (pt "L") (pt "R"))
       (blist-to-lpar-term (cdr blist)))))

(define (generate-lpar-terms n)
  (let* ((seq (generate-seq n))
         (01lists (map (lambda (f) (first f n)) seq))
         (reduced-01lists
          (list-transform-positive 01lists
            (lambda (l)
              (and (zero? (car l))
                   (not (zero? (car (last-pair l)))))))))
    (map blist-to-lpar-term reduced-01lists)))
\end{verbatim}
Now we can test \texttt{neterm-Parse} on all lpar-terms of length $l$. 
\begin{verbatim}
(define (test-parser-term parser-term . l)
  (let ((len (if (null? l) 4 (car l))))
    (map (lambda (lpar-term)
           (display "Testing on ")
           (display (term-to-string lpar-term))
           (let* ((pairterm (nt (make-term-in-app-form
                                 parser-term lpar-term)))
                  (lterm
                   (term-in-pair-form-to-left pairterm))
                  (rterm
                   (term-in-pair-form-to-right pairterm)))
             (if (and (term-in-const-form? lterm)
                      (string=?
                       "True"
                       (const-to-name
                        (term-in-const-form-to-const
                         lterm))))
                 (begin (display " Parse tree: ")
                        (display (term-to-string rterm)))
                 (display " No"))
             (newline)))
         (generate-lpar-terms len)))
    *the-non-printing-object*)
\end{verbatim}
We obtain for $l=6$
\begin{verbatim}
(test-parser-term neterm-Parse 6)
\end{verbatim}
The result is
\begin{verbatim}
Testing on L::R::R::R::R::R: No
Testing on L::L::R::R::R::R: No
Testing on L::R::L::R::R::R: No
Testing on L::L::L::R::R::R: Parse tree: C I(C I(C I I))
Testing on L::R::R::L::R::R: No
Testing on L::L::R::L::R::R: Parse tree: C I(C(C I I)I)
Testing on L::R::L::L::R::R: Parse tree: C(C I I)(C I I)
Testing on L::L::L::L::R::R: No
Testing on L::R::R::R::L::R: No
Testing on L::L::R::R::L::R: Parse tree: C(C I(C I I))I
Testing on L::R::L::R::L::R: Parse tree: C(C(C I I)I)I
Testing on L::L::L::R::L::R: No
Testing on L::R::R::L::L::R: No
Testing on L::L::R::L::L::R: No
Testing on L::R::L::L::L::R: No
Testing on L::L::L::L::L::R: No
\end{verbatim}

\section{Useful Commands}
\label{S:Commands}
\subsection{Emacs}
\label{SS:Emacs}
\begin{itemize}
\item Start Emacs: \texttt{emacs \&}
\item Leave Emacs: C-x C-c
\item Split a window in two: \texttt{C-x 2}
\item Move to another Buffer: \texttt{C-x b}
(then specify the Buffer's name)
\item Move to another window: \texttt{C-x o}
\item Load a file: \texttt{C-x C-f}
(then give a name of a file with extension \texttt{.scm})
\item Save a file: \texttt{C-x C-s}
\item Exit from the Minibuffer: \texttt{C-g}
\end{itemize}

\subsection{Scheme}
\label{SS:Scheme}
\begin{itemize}
\item Load  (Petite) Scheme:  \texttt{M-x run-petite}
\item Evaluate a Scheme expression: \texttt{C-x C-e}
\item Evaluate a region: mark the region and then \texttt{C-c C-r}
\item Kill a process: \texttt{C-c C-c} 
\item Leave the Debug: \texttt{r}
\item End a Scheme session: \texttt{(exit)}
\item Comment: \texttt{;}
\end{itemize}
\texttt{C = Control} (or \texttt{Strg}), \texttt{M = Meta} (or
\texttt{Edit} or \texttt{Esc} or \texttt{Alt}).

\subsection{Minlog}
\label{SS:Minlog}
The following is a list of commands which could be used in a
\inquotes{standard} interactive proof with \mi. Rather than explaining
the commands in detail (many of them have been demonstrated in the
above tutorial), we shall write them down, often with a short
description of their use gathered from the reference manual.  The
reader is advised to check the full details with the reference manual.

\subsubsection{Some declarations needed to start a proof}
\begin{enumerate}[]
\item \texttt{(add-tvar-name \textsl{name1} \dots)}%
\index{add-tvar-name@\texttt{add-tvar-name}}
\item \texttt{(add-algs \dots)}\index{add-algs@\texttt{add-algs}}
\item \texttt{(add-var-name \textsl{name1} \dots\ \textsl{type})}%
\index{add-var-name@\texttt{add-var-name}}
\item \texttt{(add-predconst-name \textsl{name1} \dots\ \textsl{arity})}%
\index{add-predconst-name@\texttt{add-predconst-name}}
\item \texttt{(add-pvar-name \textsl{name1} \dots\ \textsl{type})}%
\index{add-var-name@\texttt{add-pvar-name}}
\item \texttt{(add-program-constant \textsl{name} \textsl{type}
  <\textsl{rest}>)}%
\index{add-program-constant@\texttt{add-program-constant}}
\item \texttt{(add-computation-rule \textsl{lhs} \textsl{rhs})}%
\index{add-computation-rule@\texttt{add-computation-rule}}
\item \texttt{(add-rewrite-rule \textsl{lhs} \textsl{rhs})}%
\index{add-rewrite-rule@\texttt{add-rewrite-rule}}
\item \texttt{(add-global-assumption \textsl{name} \textsl{formula})}%
\index{add-global-assumption@\texttt{add-global-assumption}}
  \quad \hbox{(abbr. \texttt{aga}\index{aga@\texttt{aga}})}
\end{enumerate}
For each introduction command above there corresponds another one
having the effect of removing the item so introduced (constants,
variables,  etc). For example:

\texttt{(remove-predconst-name \textsl{name1} \dots)}%
\index{remove-predconst-name@\texttt{remove-predconst-name}}\\
There are also numerous display commands, in particular the following:
\begin{align*}
  &\texttt{(display-pconst \textsl{name1} \dots)}%
  \index{display-pconst@\texttt{display-pconst}}.
  \\
  &\texttt{(display-alg \textsl{alg-name1} \dots)}%
  \index{display-alg@\texttt{display-alg}}
  \\
  &\texttt{(display-idpc \textsl{idpc-name1} \dots)}%
  \index{display-idpc@\texttt{display-idpc}}
  \\
  &\texttt{(display-global-assumptions \textsl{string1} \dots)}%
  \index{display-global-assumptions@\texttt{display-global-assumptions}}
  \\
  &\texttt{(display-theorems \textsl{string1} \dots)}%
  \index{display-theorems@\texttt{display-theorems}}
\end{align*}
For types, terms and formulas there is a command \texttt{(pp
  \textsl{object})}\index{pp@\texttt{pp}} (for pretty-print), which
tries to insert useful line breaks.  Variants are \texttt{(ppc
  \textsl{object})}\index{ppc@\texttt{ppc}} (for pretty-print with case
display) and \texttt{(pp-subst \textsl{substitution})}%
\index{pp-subst@\texttt{pp-subst}} (for pretty-printing substitutions).

\texttt{rename-variables}%
\index{rename-variables@\texttt{rename-variables}} renames bound
variables in terms, formulas and comprehension terms.

\subsubsection{Goals}
\begin{enumerate}
\item \texttt{(set-goal \textsl{formula})}%
  \index{set-goal@\texttt{set-goal}} where
  \textsl{formula} needs to be closed (if it not so, then universal
  quantifiers will be inserted automatically).
\item \texttt{(normalize-goal .\ \textsl{ng-info})}%
  \index{normalize-goal@\texttt{normalize-goal}}
  (abbr.\ \texttt{ng}\index{ng@\texttt{ng}}) takes optional arguments
  \texttt{ng-info}.  If there are none, the goal formula and all
  hypotheses are normalized.  Otherwise exactly those among the
  hypotheses and the goal formula are normalized whose numbers (or
  names, or just \verb+#t+ for the goal formula) are listed as
  additional arguments.
\item \texttt{(display-current-goal)}%
  \index{display-current-goal@\texttt{display-current-goal}}
  (abbr.\ \texttt{dcg}\index{dcg@\texttt{dcg}}).
\end{enumerate}

\subsubsection{Generating interactive proofs}
\underline{Implication}

\texttt{(assume \textsl{x1}\dots)}\index{assume@\texttt{assume}}\\
moves the antecedent of a goal in implication form to
the hypotheses. The hypotheses,  $\textsl{x1}\dots$,
should be identified by numbers
or strings.

\texttt{(use \textsl{x})\index{use@\texttt{use}}}\\
where \textsl{x} is
\begin{itemize}
\item a number or string identifying a hypothesis from the context,
\item the string \inquotes{Truth},
\item the name of a theorem or global assumption.
\item a closed proof,
\item a formula with free variables from the context, generating a new
  goal.
\end{itemize}

\underline{Conjunction}

\texttt{(split)}\index{split@\texttt{split}}\\
expects a conjunction $A \land B$ as goal and splits it into
two new goals, $A$ and $B$.

\texttt{(use \textsl{x} . \textsl{elab-path})}\\
where \textsl{x} is as in the description of the \texttt{use}
command for implication and \textsl{elab-path} consists of \texttt{'left}
or \texttt{'right}.

\underline{Universal Quantifier}

\texttt{(assume \textsl{x1}\dots)}\index{assume@\texttt{assume}}\\
moves universally quantified variables into the context.  The variables
need to be named (by using previously declared names of
the appropriate types).

\texttt{(use \textsl{x} . \textsl{terms})}\\
where \textsl{x} is as in the case of implication and the optional
\textsl{terms} is here a list of terms.  When pattern unification
succeeds in finding appropriate instances for the quantifiers in the
goal, then these instances will be automatically inserted.  However,
one needs to explicitly provide terms for those variables that cannot
be automatically instantiated by pattern
unification.

\underline{Existential Quantifier}

\texttt{(ex-intro \textsl{term})}\index{ex-intro@\texttt{ex-intro}}\\
by this command the user provides a term to be used for
the present (existential) goal.

\texttt{(ex-elim \textsl{x})}\index{ex-elim@\texttt{ex-elim}},\\
where \textsl{x} is
\begin{itemize}
\item a number or string identifying an existential hypothesis from
  the context,
\item the name of an existential global assumption or theorem,
\item a closed proof on an existential formula,
\item an existential formula with free variables from the context,
  genera\-ting a new goal.
\end{itemize}

\underline{Classical Existential Quantifier}

\texttt{(exc-intro \textsl{terms})}\index{exc-intro@\texttt{exc-intro}}\\
this command is analogous to \texttt{(ex-intro)}, but
it is used in the case of a classical existential goal.

\texttt{(exc-elim \textsl{x})}\index{exc-elim@\texttt{exc-elim}}\\
this corresponds to \texttt{(ex-elim)} and applies to a classical
existential quantifier.

\subsubsection{Other general commands}

\texttt{(use-with \textsl{x} . \textsl{x-list})}
\index{use-with@\texttt{use-with}}\\
is a more verbose form of \texttt{use}, where the terms are not
inferred via unification, but have to be given explicitly. Here
\textsl{x} is as in \texttt{use}, and \textsl{x-list} is a list
consisting of
\begin{itemize}
\item a number or string identifying a hypothesis form the context,
\item the name of a theorem or global assumption,
\item a closed proof,
\item the string \inquotes{?}  generating a new goal,
\item \texttt{'left} or \texttt{'right},
\item a term, whose free variables are added to the context.
\end{itemize}

\texttt{(inst-with \textsl{x} . \textsl{x-list})}%
\index{inst-with@\texttt{inst-with}}\\
does for forward chaining the same as \texttt{use-with} for backward
chaining.  It adds a new hypothesis which is an instance of a selected
hypothesis or of a theorem.
Here \textsl{x} and \textsl{x-list} are as in \texttt{use-with}.

\texttt{(inst-with-to \textsl{x} . \textsl{x-list} \texttt{name-hyp})}
\index{inst-with-to@\texttt{inst-with-to}}\\
expects a string as its last argument, to name
the newly introduced instantiated hypothesis.

\texttt{(cut \textsl{A})}\index{cut@\texttt{cut}}\\
replaces the goal $B$ by the two new goals $A$ and $A \to B$, with $A
\to B$ to be proved first.  Note that the same effect can also be
produced by means of the \texttt{use} command.

\texttt{(assert \textsl{A})}\index{assert@\texttt{assert}}\\
replaces the goal $B$ by the two new goals $A$ and $A \to B$,
with $A$ to be proved first.

\texttt{(ind)}\index{ind@\texttt{ind}}\\
expects a goal $\forall_{x^\rho} A$ with $\rho$ an algebra.  If $c_1,
\dots, c_n$ are the constructors of the algebra $\rho$, then
\texttt{(ind)} will generate $n$ new goals:
\begin{equation*}
  \forall_{\vec{x}_i} (\subst{A}{x}{x_{1i}} \to \dots \to
  \subst{A}{x}{x_{ki}} \to \subst{A}{x}{c_i \vec{x}_i}).
\end{equation*}

\texttt{(simind \textsl{all-formula1}\dots)}%
\index{simind@\texttt{simind}}\\
expects a goal $\forall_{x^\rho} A$ with $\rho$ an algebra.  The user
provides other formulas to be proved simultaneously with the given
one.

\texttt{(cases)}\index{cases@\texttt{cases}}\\
expects a goal $\forall_{x^\rho} A$ with $\rho$ an algebra.  Assume
that $c_1,\dots, c_n$ are the constructors of the algebra $\rho$.
Then $n$ new (simplified) goals $\forall_{\vec{x}_i} \subst{A}{x}{c_i
  \vec{x}_i}$ are
generated.

\texttt{(simp x)}\index{simp@\texttt{simp}}\\
expects a known fact of the form $r^{\typeB}$, $\lnot r^{\typeB}$,
$t=s$ or $t \approx s$.  In case $r^{\typeB}$, the boolean term $r$ in
the goal is replaced by $T$, and in case $\lnot r^{\typeB}$ it is
replaced by $F$.  If $t=s$ (resp. $t \approx s$), the goal is written
in the form $\subst{A}{x}{t}$.  Using Compat-Rev (i.e. $\forall_{x,y}
(x=y \to P y \to P x)$) (resp. Eq-Compat-Rev (i.e. $\forall_{x,y} (x
\approx y \to P y \to P x)$)) the goal $\subst{A}{x}{t}$ is replaced
by $\subst{A}{x}{s}$, where $P$ is $\set{x}{A}$, $x$ is $t$ and $y$ is
$s$.  Here \textsl{x} is
\begin{itemize}
\item a number or string identifying a hypothesis form the context,
\item the name of a theorem or global assumption, or
\item a closed proof,
\item a formula with free variables from the context, generating a new
  goal.
\end{itemize}

\texttt{(name-hyp \textsl{i x1}})\index{name-hyp@\texttt{name-hyp}}\\
expects an index $i$ and a string.  Then a new goal is created, which
differs from the previous one only in display aspects:
the string names the $i$th hypothesis.

\texttt{(drop . x-list)}\index{drop@\texttt{drop}},\\
hides (but does not erase) the hypothesis listed in \texttt{x-list}.
If \texttt{x-list} is empty, all hypotheses are hidden.

\texttt{(by-assume \textsl{x} \textsl{y}\textsl{u})}%
\index{by-assume@\texttt{by-assume}}\\
is used when proving a goal $G$ from an existential hypothesis $ExHyp
\colon \ex y A$.  It corresponds to saying \inquotes{by $ExHyp$ assume
  we have a $y$ satisfying $A$}.  Here \textsl{x} identifies an
existential hypothesis, and we assume the variable $y$ and the kernel
$A$ (with label $u$). This command corresponds to the sequence
\texttt{(ex-elim \textsl{x})}, \texttt{(assume \textsl{y}
  \textsl{u})}, \texttt{(drop \textsl{x})}.

\texttt{(intro i .\ terms)}\index{intro@\texttt{intro}}\\
expects as goal an inductively defined predicate.  The $i$-th
introduction axiom for this predicate is applied, via \texttt{use}
(hence \texttt{terms} may have to be provided).

\texttt{(elim \textsl{idhyp})}\index{elim@\texttt{elim}}\\
Recall that $I \vec{r}$ provides (i) a type substitution, (ii) a
predicate instantiation, and (iii) the list $\vec{r}$ of argument
terms.  In \texttt{(elim \textsl{idhyp})}\index{elim@\texttt{elim}}
\textsl{idhyp} is, with an inductively defined predicate $I$,
\begin{itemize}
\item a number or string identifying a hypothesis $I \vec{r}$ form the
  context
\item the name of a global assumption or theorem $I \vec{r}$;
\item a closed proof of a formula $I \vec{r}$;
\item a formula $I \vec{r}$ with free variables from the context,
  generating a new goal.
\end{itemize}
Then the (strengthened) elimination axiom is used with $\vec{r}$ for
$\vec{x}$ and \textsl{idhyp} for $I \vec{r}$ to prove the goal
$A(\vec{r}\,)$, leaving the instantiated (with $\set {\vec{x}}
{A(\vec{x}\,)}$) clauses as new goals.

\texttt{(elim)}\index{elim@\texttt{elim}}\\
expects a goal $I \vec{r} \to A(\vec{r}\,)$.  Then the (strengthened)
clauses are generated as new goals, via \texttt{use-with}.

\texttt{(undo)}\index{undo@\texttt{undo}}
or \texttt{(undo \textsl{n})}\\
has the effect of cancelling the last step in a proof, or the
last  \textsl{n} steps, respectively.


\subsubsection{Automation and search}\quad

\texttt{(strip)}\index{strip@\texttt{strip}}\\
moves all universally quantified variables
and hypotheses of the current goal into the context.

\texttt{(strip n)}\\
does the same as \texttt{(strip)}
but only for $n$ variables or hypotheses.

\texttt{(proceed)}\index{proceed@\texttt{proceed}}\\
automatically refines the goal as far as possible as long as there is
a unique proof.  When the proof is not unique, it prompts us with the
new refined goal, and allows us to proceed in an
interactive way.

\texttt{(prop)}\index{prop@\texttt{prop}}\\
searches for a proof of the stated goal.  It is devised for
propositional logic only.

\texttt{(search \textsl{m} \textsl{(name1 m1)} \dots)}%
\index{search@\texttt{search}}\\
expects for \textsl{m} a default value of multiplicity (i.e. a
positive integer stating how often the assumptions are to be used).
Here \textsl{name1} $\dots$ are
\begin{itemize}
\item numbers or names of hypotheses from the present context or
\item names of theorems or global assumptions,
\end{itemize}
and \textsl{m1} $\dots$ indicate the multiplicities of the specific
\textsl{name1} $\dots$.
To exclude a hypothesis one can list it with multiplicity $0$.

\texttt{(auto \textsl{m} \textsl{(name1 m1)} \dots)}%
\index{auto@\texttt{auto}}\\
It can be convenient to automate (the easy cases of an) interactive
proof development by iterating \texttt{search} as long as it is
successful in finding a proof.  Then the first goal where it failed is
presented as the new goal.  \texttt{auto} takes the same arguments as
\texttt{search}.

\subsubsection{Displaying proofs objects}
There are many ways to display a proof.  We normally use
\texttt{display-proof} for a linear representation, showing the
formulas and the rules used.  We also provide a (hopefully) readable
type-free lambda expression via \texttt{proof-to-expr}, and we can add
useful information with either \texttt{proof-to-expr-with-formulas} or
\texttt{proof-to-expr-with-aconsts}.  In case the optional proof
argument is not present, the current proof is taken instead.
\begin{alignat*}{2}
  &\texttt{(display-proof .\ \textsl{opt-proof})}
  \index{display-proof@\texttt{display-proof}}
  &\quad& \text{abbreviated \texttt{dp}\index{dp@\texttt{dp}}},
  \\
  &\texttt{(display-normalized-proof .\ \textsl{opt-proof})}
  \index{display-normalized-proof@\texttt{display-normalized-proof}}
  &\quad& \text{abbreviated \texttt{dnp}\index{dnp@\texttt{dnp}}},
  \\
  &\texttt{(proof-to-expr .\ \textsl{opt-proof})},%
  \index{proof-to-expr@\texttt{proof-to-expr}}
  \\
  &\texttt{(proof-to-expr-with-formulas .\ \textsl{opt-proof})},%
  \index{proof-to-expr-with-formulas@\texttt{proof-to-expr-with-formulas}}
  \\
  &\texttt{(proof-to-expr-with-aconsts .\ \textsl{opt-proof})}.%
  \index{proof-to-expr-with-aconsts@\texttt{proof-to-expr-with-aconsts}}
\end{alignat*}
Here \texttt{display-normalized-proof} normalizes the proof first.
When in addition one wants to check the
correctness of the proof, use
\begin{align*}
  &\texttt{(check-and-display-proof .\
    \textsl{opt-proof-and-ignore-deco-flag})}%
  \index{check-and-display-proof@\texttt{check-and-display-proof}}
\end{align*}
abbreviated \texttt{cdp}\index{cdp@\texttt{cdp}}.
\texttt{ignore-deco-flag}\index{ignore-deco-flag@\texttt{ignore-deco-flag}}
is set to true as soon as the present proof argument proves a formula
of nulltype.

\subsubsection{Searching for theorems}
It is a practical problem to find existing theorems or global
assumptions relevant for the situation at hand.  To help searching
for those we provide
\begin{equation*}
  \texttt{(search-about \textsl{symbol-or-string} .\ \textsl{opt-strings})}.%
  \index{search-about@\texttt{search-about}}
\end{equation*}
It searches in \texttt{THEOREMS}\index{THEOREMS@\texttt{THEOREMS}} and
\texttt{GLOBAL-ASSUMPTIONS}%
\index{GLOBAL-ASSUMPTIONS@\texttt{GLOBAL-ASSUMPTIONS}} for all items
whose name contains each of the strings given, excluding the strings
Total Partial CompRule RewRule Sound.  It one wants to list all these as
well, take the symbol 'all as first argument.

%% \texttt{(display-proof\ . \textsl{opt-proof})}
%% \index{display-proof@\texttt{display-proof}}
%% (abbr. \texttt{dp})

%% \texttt{(display-proof-expr\ . \textsl{opt-proof})}
%% \index{display-proof-expr@\texttt{display-proof-expr}}
%% (abbr. \texttt{dpe})\\

%% \texttt{(display-eterm . \textsl{opt-proof})} (abbr. \texttt{det})\\

%% \texttt{(check-and-display-proof)}  (abbr. \texttt{cdp})\\

%% \texttt{(display-normalized-proof\ . \textsl{opt-proof})}
%% \index{display-normalized-proof@\texttt{display-normalized-proof}}
%% (abbr. \texttt{dnp}) \\

%% \texttt{(display-normalized-proof-expr\ . \textsl{opt-proof})}
%% \index{display-normalized-proof-expr@\texttt{display-normalized-proof-expr}}
%% (abbr. \texttt{dnpe})\\

%% \texttt{(display-normalized-eterm . \textsl{opt-proof})}
%% (abbr. \texttt{dnet})
\end{appendix}

\bibliography{minlog}
\bibliographystyle{amsplain}

\printindex

\end{document}
